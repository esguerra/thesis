\chapter{RNA Base Pair Steps}
\label{basepairsteps} 
\bibliographystyle{nar}
\section{Analysis (Albany Poster) and Django Webserver}
Results shown in Albany and steps part of methods paper.

This gives us the force constant matrices per base-step which are used
in the next section.

\section{Persistence Length of RNA}
A quantity commonly used to  quantify the stiffness of polymers is the
so-called persistence  length $a$. To determine this  quantity for DNA
or RNA a variety of  theoretical and experimental techniques are used.
Some  common experimental  techniques  to determine  $a$ are  Electron
Microscopy   (EM),  gel  electrophoresis,   sedimentation  velocities,
electrical  birefringence  Atomic Force  Microscopy  (AFM) ,  Magnetic
Tweezers,  and Small Angle  X-Ray Scattering  (SAXS).  For  reviews of
such  techniques  applied  to  the determination  of  RNA  persistence
length, we refer the  reader to Hagerman \cite{hagerman1998}, Abels et
al.  \cite{abels2005},  and Caliskan et  al.  \cite{caliskan2005}.  We
will  use their  results for  comparison  with those  coming from  the
"realistic"    model   developed    by    Olson   and    collaborators
\cite{olson1995} to  describe DNA. The "realistic"  model is dependent
on  high resolution  crystallographic data.   Initial  studies started
with small numbers  of data for the deformabilities  of the ten unique
base-pair steps  \cite{olson1995}. A more complete  picture applied to
the study of DNA  sequence dependent deformability became available in
1998 \cite{olson1998}.  The base-pair  step deformability data for DNA
has  been   constantly  refined  as  more  high   resolution  DNA  and
DNA-protein  structures have been  added to  the Nucleic  Acid Database
(NDB)  \cite{balasubramanian2009}.    Although  such  data   has  been
available for  DNA since 1998, it had  not been so for  RNA, until now
\cite{olson2009}.

A detailed description of the  "realistic model" along with the scheme
of the C++  code developed by Czapla and Zheng to  implement it, and a
brief account of various  definitions of persistence length and models
from which $a$ can be derived are included in Appendix~\ref{appendix4a}

\section{AMBER: Persistence Length of Base-Pair Step Patterns}
I guess it needs some input here in order to work on latex compilation.


\bibliography{biblio}

