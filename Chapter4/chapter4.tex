\chapter{RNA Base Pair Steps}
\label{basepairsteps} 
\bibliographystyle{nar}

- Base-pair steps generalities
mechanical computational scheme - refer to appendix.
unique base-pair steps
historical background from DNA work


- Collection of base-pair steps data in helical regions.
  - Yurong's bps database
  - Yurong's pyhton 3dnaparser
  - scripts to change signs, assamble unique steps, and do stats.
  - scripts to cull data
  - scripts for deformation score
  - reconstruction and RMSD calculation
  - scripts for force-constant-matrices




\section{RNA Base-Pair-Steps Database and Webframework}
A minimal MySQL database has been  created to store the mean values of
base-step   parameters,   their   standard   deviations,   and   their
corresponding  force constant  matrices. The  purpose of  creating the
database is for  later automatization of the process  of the reduction
of data, that is.

\section{Analysis of Base-Pair-Steps}
Results shown in Albany poster and steps part of methods paper.


-Potentials curve plots for steps.



After the work done for the methods paper we extract the force
constant matrices for the 10 unique canonical base-pair steps of RNA
and we use this results to determine the persistence length of RNA
using the sequence-dependent gaussian sampled model (the "realistic"
model) of Olson and collaborators.

\section{Persistence Length of RNA}
A quantity commonly used to  quantify the stiffness of polymers is the
so-called persistence  length $a$. To determine this  quantity for DNA
or RNA,  a variety of  theoretical and experimental techniques  can be
used.  Some  common experimental  techniques to determine  $a$ include
electron   microscopy   (EM),   gel   electrophoresis,   sedimentation
velocities, electrical birefringence,  atomic force microscopy (AFM) ,
magnetic  tweezers,  and small  angle  X-Ray  scattering (SAXS).   For
reviews  of  such  techniques  applied  to the  determination  of  RNA
persistence    length,    we   refer    the    reader   to    Hagerman
\cite{hagerman1997}, Abels  et al.  \cite{abels2005},  and Caliskan et
al.   \cite{caliskan2005}.  We  will compare  our  simulation results,
based on  the "realistic" model  developed by Olson  and collaborators
\cite{olson1995} to describe DNA, with their findings. The "realistic"
model is dependent  on high-resolution crystallographic data.  Initial
studies started with small numbers  of data for the deformabilities of
the  ten unique  base-pair  steps \cite{olson1995}.   A more  complete
picture applied  to the study of DNA  sequence-dependent deformability
became  available   in  1998  \cite{olson1998}.    The  base-pair-step
deformability  data  for  DNA  has  been constantly  refined  as  more
high-resolution DNA and DNA-protein  structures have been added to the
Nucleic Acid Database (NDB) \cite{balasubramanian2009}.  Although such
data has been available for DNA  since 1998, such was not the case for
RNA, until now \cite{olson2009}.

A detailed description of the  ``realistic'' model along with the scheme
of the C++  code developed by Czapla and Zheng to  implement it, and a
brief account of various  definitions of persistence length and models
from which $a$ can be derived are included in Appendix~\ref{appendix4a}

%\section{AMBER: Persistence Length of Base-Pair Step Patterns}
%I guess it needs some input here in order to work on latex compilation.


\bibliography{biblio}

