\chapter{RNA Base Pair Steps}
\label{basepairsteps} 
\bibliographystyle{nar}
A decade ago it was still not conceivable that knowledge-based
potentials could be obtained for RNA helical regions due to the small
amount of crystallographic data available. As pointed out in Chapter
2, the number of high-resolution
X-ray crystal structures of RNA has increased by two orders of
magnitude,  providing enough information to develop a dimeric model
of double-helical RNA with the 21 unique base-pair steps formed by the
canonical G·C and A·U Watson- Crick pairs and the wobble G·U
base-pair.

Using information derived from a 3.5 Å parsed subset of the
BPS (Base Pair Structure) database [2] and so-called “inverse harmonic
analysis” [3], we have derived elastic force constants for the 21
unique base-pair steps, and are using this simple scoring potential
model to simulate the fluctuations of RNA helical structures.





Motivations to study RNA dimer steps.
sequence dependent recognition of RNA by proteins



historical background from DNA work, gorin etc
data culling


The   results  obtained   from  the  analysis   of  RNA
configurational  information allow  us to  explore RNA  at  the global
level, that is, as a polymer chain, and compute the persistence length
of some RNA sequences.






unique base-pair steps
91 unique steps for the seven major populated but only 21 when
Gwobble-Uwobble is included.

- Methods Paper Results / Trends from counts
  - Data culled.
- Step Parameters. Conf Vols, RMSD
- Equipotential Curves
- Webserver
  -table counts
  -table force constants
  -potential curves
  -superimposed structures
  


- Persistence Length RNA





%\section{}

- Base-pair steps generalities



% Technical pipeline details
%- Collection of base-pair steps data in helical regions.
%  - Yurong's bps database
%  - Yurong's python 3dnaparser
%  - scripts to change signs, assamble unique steps, and do stats.
%  - scripts to cull data
%  - scripts for deformation score
%  - reconstruction and RMSD calculation
%  - scripts for force-constant-matrices



  


\section{RNA Base-Pair-Steps Database and Webframework}
A minimal MySQL database has been  created to store the mean values of
base-step   parameters,   their   standard   deviations,   and   their
corresponding  force constant  matrices. The  purpose of  creating the
database is for  later automatization of the process  of the reduction
of data, that is.

\section{Analysis of Base-Pair-Steps}


The  sequences  of  dimers  found within  the  intact,  double-helical
stretches of  high-resolution RNA structures reveal  the propensity of
the  non-canonical  pairs to  stack  against  canonical G$\cdot$C  and
A$\cdot$U  pairs rather  than  other non-canonical  pairs.  Of the  91
unique base-pair steps  that can be assembled from  the seven dominant
base-pairing  patterns, only  48 occur  in the  present  dataset, with
structural examples  for only  15 of the  45 possible  combinations of
consecutive, or tandem, non-canonical pairs (Table 5). The number of
non-canonical pairs  chemically linked  to Watson-Crick pairs  is over
six times  the number of dimers constituted  from tandem non-canonical
pairs. Some of the non-canonical  base pairs, however, occur only in
the   latter  context,   e.g.,  the   5'-UpA-3'$\cdot$5'-GpA-3'
(UA$\cdot$GA)  and 5'-ApG-3'$\cdot$5'-ApU-3'  (AG$\cdot$AU) dimers
made up of  an A$\cdot$U Hoogsteen pair linked  to a sheared G$\cdot$A
or A$\cdot$G pair. Most  of the dimers containing tandem non-canonical
pairs,  however, are  constructs  of wobble  G$\cdot$U and/or  sheared
G$\cdot$A  pairs.  As  expected  from the  large-number  of  canonical
G$\cdot$C  pairs in the  dataset, well  over a  third of  the observed
dimers are  G$\cdot$C constructs, and, surprisingly,  more than half
of these are GG$\cdot$CC steps.

The areas  of dimeric overlap, also  included in Table  5, confirm the
well-known   stacking    propensities   of   sequential   Watson-Crick
pairs. That is, the purine-pyrimidine steps show the greatest stacking
overlap  and  the pyrimidine-purine  steps  the  least.  The trend  is
similar but  the stacking is  more pronounced when a  G$\cdot$U wobble
pair  replaces  one or  both  Watson-Crick  pairs,  with the  greatest
overlap  occurring  in  self-complementary GU$\cdot$GU  steps.  Tandem
sheared G$\cdot$A pairs and dimers constituted from U$\cdot$U wobble
and sheared G$\cdot$A pairs also show appreciable overlap.  The degree
of  base  overlap, however,  does  not  seem  to govern  the  observed
populations, e.g., wobble U$\cdot$U pairs form preferentially next to
canonical  G$\cdot$C  pairs in  relatively  unstacked GU$\cdot$UC  (or
UC$\cdot$GU) dimers.

The  sequence-dependent variation of  base-pair overlap  accompanies
other  subtle changes  in dimeric  structure. For  example,  the twist
angle between  tandem Watson-Crick  G$\cdot$C pairs increases  and the
bending,  via Roll,  becomes  more negative  with  greater base-pair
overlap,  i.e., TwistCG  < TwistGG  < TwistGC  and RollCG  >  RollGG >
RollGC,  where CG,  GG, and  GC denote  the sequence  of bases  on the
leading strand of each dimer. Similar differences occur in DNA, albeit
over  a  much  wider  range  of  conformational  states.  The
sequence-dependent patterns in RNA dimeric structure, illustrated in
Fig. 4 by atomic-level  representations of tandem G$\cdot$C pairs with
the mean step parameters and phosphorus positions found in the current
dataset, are less striking  than the corresponding differences among
the dimeric arrangements of  tandem G$\cdot$U pairs. The non-canonical
pair introduces  noticeable asymmetry in  the sugar-phosphate backbone
at  the  GG$\cdot$UU  step  (Fig.  4).  The  phosphorus  atom  on  the
G-containing strand is displaced by 1.5 \AA and that on the U-containing
strand  by 1.0  \AA relative  to the  corresponding positions  in tandem
G$\cdot$C pairs. Although the phosphorus displacement is smaller and
more  symmetric,  the  twist   angle  differs  appreciably  in  the
G$\cdot$U-containing purine-pyrimidine and pyrimidine-purine dimers
compared to their Watson-Crick counterparts, with the GU$\cdot$GU step
overtwisted  by 15$\circ$  and  the UG$\cdot$UG  step undertwisted  by
-12$\circ$ compared to the GC$\cdot$GC and CG$\cdot$CG steps, respectively. Some
of these  differences reflect the computation  of base-pair parameters
with respect to  a common standard, rather than  a series of reference
frames, one  specific for each  non-canonical pair, that  minimize the
apparent ``discrepancies''.



Steps part of methods paper.




\begin{sidewaystable}[htbp]
\begin{center}
\begin{tabular}{|c|c|c|c|c|c|c|c|c|c|c|c|c|c|c|}
\hline
C$\cdot$G$_{\text{WC}}$ & G$\cdot$C$_{\text{WC}}$ & U$\cdot$A$_{\text{WC}}$ &
A$\cdot$U$_{\text{WC}}$ & U$\cdot$G$_{\text{w}}$ &
G$\cdot$U$_{\text{w}}$ & A$\cdot$G$_{\text{s}}$ &
G$\cdot$A$_{\text{s}}$ & U$\cdot$A$_{\text{H}}$ &
A$\cdot$U$_{\text{H}}$ & U$\cdot$U$_{\text{w}}$ &
A$\cdot$G$_{\text{WC}}$ & G$\cdot$A$_{\text{WC}}$ & bp$_{i}$/bp$_{i+1}$\\ 
\hline  
604 & 1335 & 747 & 574 & 77 & 192 & 66 & -- & -- & -- & 18 & 4 & 5 & G$\cdot$C$_{\text{WC}}$\\
 & 608 & 511 & 572 & 161 & 252 & 33 & -- & -- & -- & 69 & 20 & 5 & C$\cdot$G$_{\text{WC}}$\\
 &  & 97 & 249 & 20 & 45 & 7 & -- & -- & -- & 2 & -- & 3 & A$\cdot$U$_{\text{WC}}$\\
 &  &  & 126 & 48 & 79 & 6 & -- & -- & -- & 20 & 1 & 14 & U$\cdot$A$_{\text{WC}}$\\
 &  &  &  & 31 & 42 & 32 & 1 & -- & -- & 5 & -- & -- & G$\cdot$U$_{\text{w}}$\\
 &  &  &  &  & 11 & -- & -- & -- & -- & -- & 4 & 7 & U$\cdot$G$_{\text{w}}$\\
 &  &  &  &  &  & -- & 13 & -- & -- & 6 & -- & -- & G$\cdot$A$_{\text{s}}$\\ 
 &  &  &  &  &  &  & 20 & 7 & 2 & -- & -- & -- & A$\cdot$G$_{\text{s}}$\\
 &  &  &  &  &  &  &  & -- & -- & -- & -- & -- & A$\cdot$U$_{\text{H}}$\\
 &  &  &  &  &  &  &  &  & -- & -- & -- & -- & U$\cdot$A$_{\text{H}}$\\
 &  &  &  &  &  &  &  &  &  & 3 & -- & -- & U$\cdot$U$_{\text{w}}$\\
 &  &  &  &  &  &  &  &  &  &  & -- & 1 & G$\cdot$A$_{\text{WC}}$\\
 &  &  &  &  &  &  &  &  &  &  &  & -- & A$\cdot$G$_{\text{WC}}$\\
\hline
\end{tabular}
\caption{Unique base-pair steps parameters counts and overlap values
  in RNA helical regions.}
\label{tab:91steps}
\end{center}
\end{sidewaystable}



Results from Albany

-Potentials curve plots for steps.



After the work done for the methods paper we extract the force
constant matrices for the 10 unique canonical base-pair steps of RNA
and we use this results to determine the persistence length of RNA
using the sequence-dependent gaussian sampled model (the "realistic"
model) of Olson and collaborators.




\section{Persistence Length of RNA}
A quantity commonly used to  quantify the stiffness of polymers is the
so-called persistence  length $a$. To determine this  quantity for DNA
or RNA,  a variety of  theoretical and experimental techniques  can be
used.  Some  common experimental  techniques to determine  $a$ include
electron   microscopy   (EM),   gel   electrophoresis,   sedimentation
velocities, electrical birefringence,  atomic force microscopy (AFM) ,
magnetic  tweezers,  and small  angle  X-Ray  scattering (SAXS).   For
reviews  of  such  techniques  applied  to the  determination  of  RNA
persistence    length,    we   refer    the    reader   to    Hagerman
\cite{hagerman1997}, Abels  et al.  \cite{abels2005},  and Caliskan et
al.   \cite{caliskan2005}.  We  will compare  our  simulation results,
based on  the "realistic" model  developed by Olson  and collaborators
\cite{olson1995} to describe DNA, with their findings. The "realistic"
model is dependent  on high-resolution crystallographic data.  Initial
studies started with small numbers  of data for the deformabilities of
the  ten unique  base-pair  steps \cite{olson1995}.   A more  complete
picture applied  to the study of DNA  sequence-dependent deformability
became  available   in  1998  \cite{olson1998}.    The  base-pair-step
deformability  data  for  DNA  has  been constantly  refined  as  more
high-resolution DNA and DNA-protein  structures have been added to the
Nucleic Acid Database (NDB) \cite{balasubramanian2009}.  Although such
data has been available for DNA  since 1998, such was not the case for
RNA, until now \cite{olson2009}.

A detailed description of the  ``realistic'' model along with the scheme
of the C++  code developed by Czapla and Zheng to  implement it, and a
brief account of various  definitions of persistence length and models
from which $a$ can be derived are included in Appendix~\ref{appendix4a}

%\section{AMBER: Persistence Length of Base-Pair Step Patterns}
%I guess it needs some input here in order to work on latex compilation.


\bibliography{biblio}

