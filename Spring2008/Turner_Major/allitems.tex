\title{Molecular Biophysics Seminar Suggested Readings (Spring 2008)}
\author{Dr. Wilma K. Olson\thanks{With help from Mauricio Esguerra}}
\maketitle
\bibliographystyle{nar}
\label{readings} 


\section*{\underline{Douglas H. Turner}}
\subsection*{Using Constraints to Determine RNA Secondary and Local 3D Structure}

\begin{thebibliography}{1}
\bibitem[1]{chemcons} David H. Mathews and Matthew D. Disney and
Jessica L. Childs and Susan J. Schroeder and Michael Zuker and
Douglas H. Turner (2004) {\em Incorporating Chemical Modification Constraints into
a Dynamic Programming Algorithm for Prediction
of RNA Secondary Structure.} Proceedings of the National Academy of
Science, 101, 7287-7292.

\bibitem[2]{arraycons}Elzbieta Kierzek and Ryszard Kierzek and Walter N. Moss and 
Shawn M. Christensen and Thomas H. Eickbush and Douglas H. Turner {\em
Isoenergetic penta- and hexanucleotide microarray probing and
chemical mapping provide a secondary structure model for an RNA
element orchestrating R2 retrotransposon protein function} Nucleic
Acids Research http://dx.doi.org/10.1093/nar/gkm1085  .

Gang Chen et al., J. Phys. Chem. B 111, 6718-6727 (2007)








\bibitem[4]{general} David H. Mathews and Douglas H. Turner (2006) 
{\em
    Prediction of RNA Secondary Structure by Free Energy
    Minimization.} Current Opinion in Structural Biology., 16, 270-278.

\bibitem[3]{arrays} Elzbieta Kierzek and Ryszard Kierzek and Douglas
  H. Turner and irina E. Catrina (2006) {\em Facilitating RNA
    Structure Prediction with Microarrays.} Biochemistry., 45, 581-593.
\end{thebibliography}

\newpage

\section*{\underline{Fran\c cois Major}}
\subsection*{Theory and Application of a Novel RNA Folding Approach Based on
Nucleotide Cyclic Motifs}

\begin{thebibliography}{1}
\bibitem[1]{ncm} Lemieux, S.  and Major, F.  (2006) {\em Automated Extraction and
Classification of RNA Tertiary  Structure Cyclic Motifs.} Nucleic Acids
Res., 34, 2340-2346.

\bibitem[2]{3Dmotif} St-Onge,  K., Thibault, P., Hamel,  S. and Major,
  F. (2007) {\em Modeling
RNA Tertiary  Structure Motifs  by Graph-grammars.} Nucleic  Acids Res.,
35, 1726-1736.

\bibitem[3]{software} Major, F.  (2003) {\em Building Three-Dimensional Ribonucleic Acid
Structures.} IEEE Comp Science Eng., 5, 44-53.
\end{thebibliography}

\noindent
\textbf{ - Abstract from talk presented on October 31st at the Institute
  for Mathematics and its  Applications at the University of Minnesota
  at Minneapolis.}
\\

\noindent
We change the classical  rationale underlying RNA structure prediction
by  incorporating  the  contributions  of  the  non-Watson-Crick  base
pairs. To do  so, we define a new  first-order object for representing
nucleotide relationships in structured  RNAs, which we call nucleotide
cyclic  motif (NCM)  (1). In  comparison  to the  classical stacks  of
Watson-Crick base pairs, the properties that make NCMs
appealing for structure determination are  the facts that: i) the same
algorithm can be employed  for predicting secondary, tertiary, and 3-D
structures; ii)  the RNA structural motifs  are either made  of one or
more NCMs (2);  iii) the NCMs embrace indistinctly  both canonical and
non-canonical base pairs;  and, iv) the NCMs precisely
designate how  any nucleotide in a  sequence relates to  the others. A
structure   generator  and  scoring   function  has   been  developed:
MC-Fold. We show  how MC-Fold, combined to MC-Sym  (3), builds RNA 3-D
structures from  sequence data and, combined to  MC-Cons, clusters and
aligns RNA  family sequences. We  show how low-resolution data  can be
incorporated in  the modeling to reach conformational  states that are
difficult to access by sequence data alone.





