%  Title and other sections that come before the body  of the document
\phd
%\jointumdnj
\title{\textbf{RNA Structure Analysis via the Rigid Block Model}}
\author{Mauricio Esguerra Neira}
\campus{New Brunswick}
\program{Chemistry and Chemical Biology}
\director{Wilma K. Olson}

%% Gives the number of lines for comittee signatures.
\approvals{4}


% \copyrightpage % Do you want copyright protection?
\submissionmonth{May}   % only May, October or January
\submissionyear{2010}
%\today
%\figurespage
%\draft
%%% Abstract
\abstract{RNA structure is at the forefront of our understanding of
the origin of life, and the mechanisms of life regulation and
control. RNA plays a primordial role in some viruses.
Our knowledge of the importance of RNA in cellular regulation is
relatively new, and this knowledge, along with the detailed structural
elucidation of the transcription machine, the ribosome, has propelled
interest in understanding RNA to a level which starts to closely
resemble that given to proteins and DNA.

In the process of progressively understanding the landscape of
functionality of such a complex polymer as RNA, one practical task
left to the structural chemist is to understand the details of how
structure relates to large-scale polymer processes. With this in mind
the fundamental problems which fuel the work described in this thesis
are those of the conformations which RNA's assume in nature,
and the aim to understand how RNA folds.

The RNA folding problem can be understood as a mechanical
problem. Therefore efforts to determine its solution are not foreign
to the use of statistical mechanical methods
combined with detailed knowledge of atomic level structure. Such
methodology is mainly used in this work in a long-term effort
to understand the intrinsic structural features of RNA, and how
they might relate to its folding.}


%%% Acknowledgments
\acknowledgements{I would first like to give a special thanks to
Dr. Yurong Xin, whose patience, help, and collaboration since the very
beginning of my joining of the Olson lab have been fundamental for
the development of this work. Also I would like to thank comments and
help with the persistence lenght code to doctors Luke Czapla, and
Guohui Zheng who were very kind and prompt in answering very technical
questions concerning the code. I would like to thank Dr. Olson's
extreme patience and room for freedom on carrying out this
research. Finally I thank all colleagues at the Olson lab.\\

I would like to dedicate this thesis to David and Stella Case, without
them these words would not exist.}

\quotes{
\textit{As a thing among things, each thing is equally insignificant; as
a world each one equally significant.} 

\textit{If I have been contemplating the stove, and then am told; but now all
you know is the stove, my result does indeed sound trivial. For this
represents the matter as if I had studied the stove as one among the
many, many things in the world. But if I was contemplating the stove,
it was my world, and everything else colorless by contrast with it ...}

\textit{For it is equally possible to take the bare present image as the
worthless momentary picture in the whole temporal world, and as the
true world among shadows.}

\begin{flushright}
\textbf{Ludwig Wittgenstein}
\end{flushright}

\vspace{2 cm}

\textit{As a molecule among molecules, each molecule is equally
insignificant; as a world each one equally significant.} 

\textit{If I have been contemplating RNA, and then am told; but now all
you know is RNA, my result does indeed sound trivial. For this
represents the matter as if I had studied RNA as one among the
many, many molecules in the world. But if I was contemplating RNA,
it was my world, and everything else colorless by contrast with it ...}

\textit{For it is equally possible to take the bare present image as the
worthless momentary picture in the whole temporal world, and as the
true world among shadows.}

\begin{flushright}
\textbf{Anonymous Chemist}
\end{flushright}

}

%\end{singlespace}

\figurespage
\tablespage
