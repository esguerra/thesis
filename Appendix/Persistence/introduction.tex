\part{Introduction}
Biopolymers can be either rigid or flexible. 

They can be classified  according to whether their persistence lenghts
are greater, smaller, or similar to the total length of the polymer.

\begin{table}[htbp]
\begin{center}  
\begin{tabular}{c|c|c}
\hline
Rigid Rod       & Rigid          &        $lp > L$   \\
Gaussian chain  & Flexible       &        $lp < L$   \\
Worm-like chain & Semi-flexible  &    $lp \approx L$ \\
\hline
\end{tabular}
\end{center}
\end{table}

Rigid biopolymers:
actin, microtubules

Flexible biopolymers:


Semi-flexible:

If the persistence lenght is of the same order of the length of the
polymer, then the polymer is classified as  semi-flexible

\section{Persistence Lenght Definitions}

``bend-persistence length: 
A length scale beyond which the elastic cost of bending is totally
negligible''

``In a randomly shaken rod any particular point in the rod will be
pointing in a random direction, but nearby points will be pointing in
roughly the same direction, that is, this nearby points are
persistent. Points farther away than the bend-persistence length are
said to be uncorrelated.''

there is also twist-persistence length.
that is, there is, a rubber rod not only resists bending but also twisting.

"length at which the orientation of the sequential bonds which make
up a polymer chain, stop being correlated. That is, if you have just
two bonds, they will be correlated, which is the case in most
molecules, but, in polymers, you have a long chain of sequential
bonds. At some length, bonds will become uncorreleted, but up to that
lenght they were correlated, this is what is meant by persistence
length, and, in this context it's obvious that is an exclusive
property of polymers."



\begin{table}[htbp]
\begin{center}
\begin{tabular}{|l|l|l|l|}
\hline 
polymer & lp(\AA) & T (Kelvin) & Phase \\ \hline
DNA     & 500     &  273       & water \\ \hline
\end{tabular}
\end{center}
\end{table}



\section{end-to-end}
The end-to-end vector  $r$ is the vector which connects  the ends of a
chain.  It  can be defined  as the sum  of the vectors connecting the
monomer units in a chain. These connecting vectors are sometimes called
virtual  bond vectors  $l$. From  the end-to-end  vector  the quantity
which is usually of interest is it's magnitude.

\begin{gather}
\label{eq:end2end}
r = \sum_{i=1}^{n} l_{i}\\
r^2 = r \cdot r = \sum_{i,j}l_{i} \cdot l_{j}
\end{gather}
Equation \ref{eq:end2end}, can also be written:
\begin{gather}
r^2 = \sum_{i}l_{i}^{2} + 2 \sum_{i\neq j} l_{i} \cdot l_{j}
\end{gather}  

To  describe a  polymer  it's  necessary to  think  about the  various
conformations it can adopt  due to its flexibility, therefore, it
is important  to think of polymer  related quantities in  terms of the
average of their possible conformations. For the end-to-end vector the
average of  its values  is denoted  as $<r>$, and  the average  of its
norm, also called the second moment of the end-to-end distribution, is
denoted by $<r^2>$:
\begin{gather}
\label{eq:secmom}  
<r^2>=\sum_{i}<l_{i}^2> + 2\sum_{i<j}<l_{i} \cdot l_{j}>
\end{gather}  

When there is no correlation between succesive bonds we can write:

\begin{gather}
\label{eq:nocorr}
<l_{i} \cdot l_{j}> = 0
\end{gather}

And equation \ref{eq:secmom} reduces to having only the bond self
correlation term:

\begin{gather}
<r^2> = \sum_{i}<l_{i}^2> = n<l^2>
\end{gather}  

This equation is used to describe a so-called freely-jointed chain.
