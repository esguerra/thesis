\part{Introduction}
Biopolymers can be either rigid or flexible. 

They are classified according to whether their persistence lenghts are
greater, smaller, or similar to the total lenght of the polymer.

Flexible          $lp < L$\\
Rigid             $lp > L$\\
Semi-flexible     $lp \approx L$\\

Rigid biopolymers:
actin, microtubules

Flexible biopolymers:


Semi-flexible:

If the persistence lenght is of the same order of the length of the
polymer, then the polymer is classified as  semi-flexible

\section{Persistence Lenght Definitions}

``bend-persistence length: 
A length scale beyond which the elastic cost of bending is totally
negligible''

``In a randomly shaken rod any particular point in the rod will be
pointing in a random direction, but nearby points will be pointing in
roughly the same direction, that is, this nearby points are
persistent. Points farther away than the bend-persistence length are
said to be uncorrelated.''


there is also twist-persistence length.
that is, there is, a rubber rod not only resists bending but also twisting.



\begin{tabular}{|l|l|l|l|}
\hline 
polymer & lp(\AA) & T (Kelvin) & Phase \\ \hline
DNA     & 500     &  273       & water \\ \hline
\end{tabular}
