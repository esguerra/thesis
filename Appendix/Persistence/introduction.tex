\part{Introduction}
\bibliographystyle{nar}
Nucleic Acids and other polymers can be understood as mechanical
objects \cite{marko2003, nelson2004} and therefore engineering
approaches can be used for their understanding. Usually the methods
followed by the engineering approach neglect details which can be
intrinsic to the local chemistry of the subunits which make up the
polymer. Dr. Olson and collaborators have suggested a formal approach
refered to as the "realistic" model, which takes into account details
at the sequence level for the case of nucleic acids. This model is
dependent on a knowledge-based approach, where, force constant analogs
are derived from crystallographic data using a so called covariation,
or go model \cite{go1974}.





Continuous vs discrete models.





The previous is not exactly true, we are using a completely physical
model, the only thing is that we are linking different parts, that is,
we are assuming that the connected parts have springs with different
constants, that't it. It is as if we were connecting pieces of steel,
with pieces of rubber, let's say.


Global properties of DNA as a polymer do not depend so much in
structural detail, but local properties are definetely described at
the chemical-molecular-atomic level.

The realistic model, in a way, bridges the gap between non-sequence
dependent elasticity theory models, and sequence dependent
molecular-bond (can be virtual-bond) based models.


that is, like rigid rods, say a steel bar, or like flexible rods, say
a rubber tube.

The "realistic model" is general an can be simplified to other models,
for example to the freely rotating model, just by making the twisting
constant zero, that is, the polymer will not be elastic when twisted.
when stretching constan 



To quantify the rigidity of a polymer

Biopolymers can be either rigid or flexible. 

They can be classified  according to whether their persistence lengths ($a$)
are greater, smaller, or similar to the total length of the polymer ($L$).

\begin{table}[htbp]
\begin{center}  
\begin{tabular}{c|c|c}
\hline
Model Type      & Polymer Characteristic & a to L relation\\ \hline
Rigid Rod       & Rigid          &        $a > L$   \\
Gaussian chain  & Flexible       &        $a < L$   \\
Worm-like chain & Semi-flexible  &    $a \approx L$ \\
\hline
\end{tabular}
\end{center}
\end{table}

Worm-like-chain = Porod-Kratky = Freely Rotating Chain in limit l=0
and n=infinity
\\

Rigid biopolymers:
actin, microtubules
\\

Flexible biopolymers:
\\

Semi-flexible biopolymers:
High force extension DNA.
\\

If the persistence length is of the same order of the length of the
polymer, then the polymer is classified as  semi-flexible


\begin{table}[htbp]
\begin{center}  
\begin{tabular}{c|c}
\hline
Polymer       & $a$ (nm)   \\ \hline
$\alpha$-helix & 80-100\\
coiled-coil & 150-300\\
Ideal DNA  &  51  \\
Ideal RNA & 70-80 \\
\hline
\end{tabular}
\caption{Persistence lengths for some biopolymers with filament structures.}
\end{center}
\end{table}

Think about the "energy" based perspective of Nicolas, and the
stochastic based perspective of Flory and others.




\section{Persistence Lenght Definitions}

``\textbf{Bend-persistence length}:\\ 
A length scale beyond which the elastic cost of bending is totally
negligible''

``In a randomly shaken rod any particular point in the rod will be
pointing in a random direction, but nearby points will be pointing in
roughly the same direction, that is, this nearby points are
persistent. Points farther away than the bend-persistence length are
said to be uncorrelated.''

there is also twist-persistence length.
that is, there is, a rubber rod not only resists bending but also twisting.

"length at which the orientation of the sequential bonds which make
up a polymer chain, stop being correlated. That is, if you have just
two bonds, they will be correlated, which is the case in most
molecules, but, in polymers, you have a long chain of sequential
bonds. At some length, bonds will become uncorreleted, but up to that
lenght they were correlated, this is what is meant by persistence
length, and, in this context it's obvious that is an exclusive
property of polymers."

"the average sum of the projections of all bonds $ j \geq i$  in an 
indefinetely long chain."


\begin{table}[htbp]
\begin{center}
\begin{tabular}{|l|l|l|l|}
\hline 
polymer & lp(\AA) & T (Kelvin) & Phase \\ \hline
DNA     & 500     &  273       & water \\ \hline
\end{tabular}
\end{center}
\end{table}



\section{end-to-end}
The end-to-end vector  $r$ is the vector which connects  the ends of a 
polymer chain.  It  can be defined  as the sum  of the vectors connecting the
monomer units in a chain. These connecting vectors are sometimes called
virtual  bond vectors $l$.  

From  the end-to-end  vector  the quantity
which is usually of interest is it's magnitude.

\begin{gather}
\label{eq:end2end}
r = \sum_{i=1}^{n} l_{i}\\
r^2 = r \cdot r = \sum_{i,j}l_{i} \cdot l_{j}
\end{gather}
Equation \ref{eq:end2end}, can also be written:
\begin{gather}
r^2 = \sum_{i}l_{i}^{2} + 2 \sum_{i\neq j} l_{i} \cdot l_{j}
\end{gather}  

To  describe a  polymer  it's  necessary to  think  about the  various
conformations it can adopt  due to its flexibility, therefore, it
is important  to think of polymer  related quantities in  terms of the
average of their possible conformations. For the end-to-end vector the
average of  its values  is denoted  as $<r>$, and  the average  of its
norm, also called the second moment of the end-to-end distribution, is
denoted by $<r^2>$:
\begin{gather}
\label{eq:secmom}  
<r^2>=\sum_{i}<l_{i}^2> + 2\sum_{i<j}<l_{i} \cdot l_{j}>
\end{gather}  

When there is no correlation between succesive bonds we can write:

\begin{gather}
\label{eq:nocorr}
<l_{i} \cdot l_{j}> = 0
\end{gather}

And equation \ref{eq:secmom} reduces to having only the bond self
correlation term:

\begin{gather}
<r^2> = \sum_{i}<l_{i}^2> = n<l^2>
\end{gather}  

This equation is used to describe a so-called freely-jointed chain.


\section{Models}

Nelson in book says:

\begin{gather}
dE={1}{2}\kappa_{B}T[A\beta^2+Bu^2+C\omega^2+2Du\omega]ds
\end{gather}  

A$\kappa \beta$ T = Bend stiffness
B$\kappa \beta$ T = Stretch stiffness
C$\kappa \beta$ T = Twist stiffness
D$\kappa \beta$ T = Twist-stretch coupling

If only the bend stiffness survives then the model is called an
inextensible model, also Porod-Kratky, or WLC.

\subsection{Kuhn - Freely Jointed Chain (FJC)}

\subsection{Porod-Kratky - Worm Like Chain (WLC)}

\subsection{Olson - Realistic}

The Hamiltonian for a \cite{czapla2009}





\section{Suggested Reads}

From Equilibrium Statistics of Plischke and Bergersen they suggest to
read:
Des Cloiseaoux and Janik ()
Rubinstein and Colby (Polymer Physics)
\bibliography{biblio}
