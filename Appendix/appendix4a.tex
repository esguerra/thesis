\chapter{Persistence Length}
\label{appendix4a}
\bibliographystyle{nar}
Nucleic Acids and other polymers, due to their size, can be understood
as  mechanical  objects  \cite{marko2003,  nelson2004}  and  therefore
engineering approaches  are often  used for their  understanding.  The
methods which consider the polymer as a long continuous rod, are known
as continuum elastic  theory.  This type of model  leaves little space
for taking into  account the nature of the subunits  which make up the
polymer. That is,  it is mainly applicable to  homopolymers made up of
identical  subunits  with   limited  bending,  twisting,  and  stretch
motion. In nucleic  acids this is not necesarily the  case, and a more
general approach  is needed  to take into  account the  possibility of
having different subunits and subunits with different levels of motion
in   the  polymer.    Olson   and  collaborators   have  developed   a
sequence-dependent  model,  refered  to  as  the  ``realistic''  model
\cite{olson1993}  to treat  long, fluctuating  DNA helices.  The model
uses a harmonic  approximation to treat the motion  of base-pair steps
and  uses   force-constants  and   rest  states  derived   from  X-Ray
crystallographic   data   from  the   Nucleic   Acid  Database   (NDB)
\cite{go1976,  olson1998}.  Within  the context  of  the ``realistic''
model,  Czapla et  al.   \cite{czapla2006} have  developed a  Gaussian
sampling  methodology  to generate  random  chains configurations  and
adapted matrix  methods to compute global polymer  properties like the
persistence   length  from   Flory  \cite{flory1969}   and   Olson  et
al. \cite{maroun1988a, marky1994a}

In what follows we summarize definitions of persistence length and how
it is computed using different models.
%The "realistic model" can be simplified to other  models, for example
%to the  freely rotating model, just by  making the  twisting constant
%zero, that is, the polymer will not be elastic when twisted.

\section{Persistence Length Definition}
%%Perhaps it would be good to say, "parallel and equivalent" perspect.
In  general there  are two  parallel perspectives  used to  define the
persistence length of polymers. One  of these has a more intrinsically
mathematical, or  physical flavor, in which the  persistence length is
understood as  the resistance  to deformation of  a curve in  space (a
mathematical object), or a thin rod (a physical object).  The other is
a stochastic definition, where the persistence length is understood as
``a measure of the distance over which the direction of the polymer is
maintained'' \cite{kratky1949}.  In  both cases the persistence length
can be understood as a measure of polymer stiffness.

Within the context of the ``mathematical physics'' definition we cite Marko's
and Nelson's definitions:

\begin{quotation}
``Classical elasticity tells us that a thin, straight rod that is bent
into an arc has a bending energy $E=Bl/2R^2$, where $B$ is the bending
elastic constant of the  rod, $l$ is the length of the  rod and $R$ is
the radius  of arc. Setting  $R=l$ gives us  the energy of a  1 radian
bend along the rod, and solving  for when $E \sim K_{B}T$ gives us the
length  of  rod along  which  a thermally  excited  bend  of 1  radian
typically occurs:  $l \sim B/K_{B}T$.  This is  called the persistence
length...'' John F. Marko and Simona Cocco, Physics World, March 2003
\end{quotation}

\begin{quotation}
``In the elastic rod model of a polymer, the elastic energy of a short
segment       of       rod       is       $dE=\frac{1}{2}       K_{B}T
[A\beta^2+Bu^2+C\omega^2+2Du\omega]  ds$.  Here  $AK_{B}T$, $CK_{B}T$,
$BK_{B}T$,  and $DK_{B}T$  are  the bend  stiffness, twist  stiffness,
stretch stiffness, and twist-stretch coupling, and $ds$ is the length of
the segment.   (The quantities $A$  and $C$ are  also called the  bend and
twist  persistence  lengths.)''   Philip Nelson,  Biological  Physics:
Energy, Information, Life, 2004.
\end{quotation}

These definitions coming from the point of view of physicists refer to
an ideal  continuous thin rod.  Marko, refers  to the bend-persistence
length,  whereas Nelson's  definition uses  a more  general  notion of
persistence length which includes, besides the bend persistence length,
twist and stretch persistence lengths.

From the ``stochastic'' perspective we cite Flory's definition:

\begin{quotation}
Persistence  length is  ``the average  sum of  the projections  of all
bonds $ j \geq i$ on bond $i$ in an indefinetely long chain.  The bond
$i$ is taken to be remote from either end of the chain, i.e., $1 \ll i
\ll   n$''.   Paul   J.   Flory,  Statistical   Mechanics   of   Chain
Molecules. 1969
\end{quotation}


This  second perspective  is more  familiar to  the chemist,  since it
assumes  some type  of  bonded connectivity  between polymeric  units,
where the bond can be either ``real'' or ``virtual''.

Both  perspectives  are  analogs  of  one  another  since  the  latter
definition can  be made to appear  like the former in  the limit where
the length  of the bonds connecting  the monomeric units  is zero, and
the  number  of  monomers  reaches  a  very  large  number,  formally,
infinity.

When talking about persistence length  it is usually difficult to have
a good idea of  the meaning of the quantity by itself.  That is, if we
are told  that the persistence length  of DNA is, say,  530 \AA~ under
some specific concentration and  temperature conditions, we don't know
what this  is telling us about  its stiffness since we  don't know any
other  standard  values  to relate  this  one  to.  To give  a  better
understanding of the meaning of  the values of the persistence length,
we have collected in  Table~\ref{tab:perval} values of the persistence
length    for   various    filamentous   biopolymers.     Looking   at
Table~\ref{tab:perval}  it  is  clear  that  B-DNA is  quite  a  stiff
biopolymer if compared to poly-glycine, or poly-alanine.

\begin{table}[htbp]
\begin{center}
\begin{threeparttable}
\begin{tabular}{l|r|c}
\hline
Polymer               & $a$ (nm) & Citation  \\ \hline
Polymethylene         & 0.6      & Flory\tnote{a}      \\  
Polystyrene           & 0.9      & Flory\tnote{a}      \\
Polyglycine           & 0.6      & Flory \tnote{b}     \\
Poly-L-alanine        & 2        & Flory \tnote{b}     \\
Poly-L-proline        & 22       & Cantor and Schimel \cite{cantor1980} \\
B-DNA                 & 53       & Rivetti \cite{rivetti1996}     \\
%A-RNA                 & 70-80    & Hagerman  \cite{hagerman1997}  \\
A-RNA                 & 62-64    & Abels     \cite{abels2005}     \\
$\alpha$-helix        & 80-100   & Lakkaraju \cite{lakkaraju2009} \\
Coiled-coil           & 150-300  & Lakkaraju \cite{lakkaraju2009} \\
Neurofilament         & 500      & Nelson    \cite{nelson2004}    \\
Intermediate Filament & 1000     & Lakkaraju \cite{lakkaraju2009} \\
F-Actin               & 17000    & Lakkaraju \cite{lakkaraju2009} \\
Microtubule           & 5200000  & Lakkaraju \cite{lakkaraju2009} \\
\hline
\end{tabular}
\begin{tablenotes}
\item [a] Computed using  characteristic ratios $C_{\infty}$ reported in Table 1
  of Flory's book \cite{flory1969} using a C-C bond length $\nu = 1.54$
  \AA. The derivation and formula for $C_{\infty}$ can be found in section \ref{sec:fjc}.
\item [b] Computed using  characteristic ratios reported in Table 3
  of Flory's book \cite{flory1969} using a virtual bond length $\nu = 3.80$ \AA~
between sequential c-alpha atoms C${\alpha}_{i}$-C${\alpha}_{i+1}$.

\end{tablenotes}
\end{threeparttable}
\caption{Persistence lengths for common polymers, and biopolymers with
  filament structures.}
\label{tab:perval}
\end{center}
\end{table}

%Using  the  persistence  length  $a$  one  can  classify  biopolymers
%according to  their stiffness into rigid,  flexible, and semi-flexible
%polymers as shown in Table~\ref{tab:pers}.
%\begin{table}[H]
%\begin{center}  
%\begin{tabular}{c|c|c|c}
%\hline
%Model           & Polymer Type & $a$ to $L$ relation & Examples\\ \hline
%Rigid Rod       & Rigid          &  $a \gg L$     &  Actin, Microtubules\\
%Gaussian chain  & Flexible       &  $a \ll L$     &  \\
%Worm-like chain & Semi-flexible  &  $a \approx L$ &  High Force Extension DNA\\
%\hline
%\end{tabular}
%\label{tab:pers}
%\caption{Relation of persistence length ($a$) to contour length ($L$)
%  and polymer classification.}
%\end{center}
%\end{table}

There are other definitions of persistence length which arise when one
wants   to   take   into   account   the   electrostatic   nature   of
polyelectrolytes, for example; Skolnick and Fixman \cite{skolnick1977}
proposed  an  electrostatic  persistence  length  as a  result  of  an
extension  of the  so-called  Porod-Kratky chain  to include  charges.
Another  approach is  that of  Manning, which  proposed an  ideal case
where  the charge  of the  polyelectrolyte (DNA)  would  be completely
neutralized\footnote{Manning defines DNA* as the null charge isomer of
  charged DNA.}  and for  such neutralized molecule a null persistence
length  is  defined  \cite{manning2006}.   Yet another  definition  of
persistence length by Trifonov et al. \cite{trifonov1987} appears when
``static''  and  ``dynamic''   components  are  considered.   In  such
definition the ``static'' components come from static bends like those
produced by phased A-tracts in DNA.

\section{Freely Jointed Chain (FJC)}
\label{sec:fjc}
The end-to-end  vector $\mathbf{r}$\footnote{Note  that we will
  be using bold  face letters to denote vectors.}  is the vector which
connects the ends of a polymer chain and isdefined as the sum of
the  vectors  connecting  the  monomeric  units in  a  chain.  These
connecting   vectors   can   be   either  ``real-bond''   vectors   or
``virtual-bond''  vectors  and   are  denoted  by  $\mathbf{l}$.   The
magnitude  of  the  end-to-end  vector  is  usually  the  quantity  of
interest as given by equation \ref{eq:end2end}.

\begin{gather}
\label{eq:e2e}
\mathbf{r} = \sum_{i=1}^{n} \mathbf{l}_{i}\\
\label{eq:end2end}
r = \sqrt{\mathbf{r} \cdot \mathbf{r}}
  = \sqrt{\sum_{i,j}\mathbf{l}_{i} \cdot \mathbf{l}_{j}}
\end{gather}

To distinguish between the scalar product of bond vectors with
themselves and with their neighbors equation \ref{eq:end2end} is
rewritten as:

\begin{gather}
r^2 = \sum_{i=1}^{n}l_{i}^{2} + 2 \sum_{i\neq j}^{n}
\mathbf{l}_{i} \cdot \mathbf{l}_{j}
\end{gather}  

The previous description takes into account only a single case for a chain
conformation, but to describe a  polymer it's necessary to think about
the   various  conformations   the  chain   can  adopt   due   to  its
flexibility. Therefore,  it's  important to think  of polymer-related
quantities   in   terms   of   the   average   properties   over   all
conformations (i.e.,  an ensemble). The  average of the ensemble  of 
end-to-end vectors  is denoted by $\left< \mathbf{r}  \right>$, and the
average of its  squared value, also known as the  second moment of the
end-to-end  distribution is denoted  by $\left<  r^2 \right>$  and is
given by the expression:

\begin{gather}
\label{eq:secmom}  
\left<r^2\right>=\sum_{i}^{n}\left<l_{i}^2\right> +
2\sum_{i<j}^{n}\left<\mathbf{l}_{i} \cdot \mathbf{l}_{j}\right>
\end{gather}  

When there is no correlation between succesive bonds the average
scalar product between them vanishes.

\begin{gather}
\label{eq:nocorr}
\left<\mathbf{l}_{i} \cdot \mathbf{l}_{j}\right> = 0
\end{gather}

A geometrical interpretation of this  is that every bond is allowed to
rotate freely around its immediate neighbors.  What is meant precisely
by  ``freely'', is  that bond  angles  can randomly  assume any  value
between  0  and  360  degrees,   and  that  there  are  no  bond-angle
constraints whatsoever.   When the number  of conformations approaches
infinity the average cosine of  bond angles will be zero and therefore
it's  scalar product.   Equation \ref{eq:secmom} keeps  only the
bond auto-correlation term:

\begin{gather}
\label{eq:fjc}  
\left<r^2 \right> = \sum_{i=1}^{n}\left<l_{i}^2\right> = nl^2
\end{gather}  

This equation is used to describe a so-called freely-jointed chain
(FJC), which also corresponds to a 3D random walk.

A quantity to quickly check if a chain behaves as a FJC is the
characteristic ratio:

\begin{gather}
C_{n}=\frac{\left<r^{2}\right>}{nl^{2}}
\end{gather}

Which will be unity for a FJC. In most cases as $n \to \infty$ the
characteristic ratio is greater than one.

\section{Realistic chains}
In general polymer chains will have correlation between neighboring
bonds and so the second term in the right hand side of equation
\ref{eq:secmom} has to be taken into account.
Using equation \ref{eq:fjc} and expanding the second right
hand term of equation \ref{eq:secmom} we have:

\begin{gather}
\label{eq:char}  
<r^{2}> = nl^{2} + 2 \sum_{j=2}^{n}
\left<  \mathbf{l}_{1} \cdot \mathbf{l}_{j}\right> +
2 \sum_{j=i+1}^{n}\sum_{i=2}^{n}
\left<  \mathbf{l}_{i} \cdot \mathbf{l}_{j}\right>
\end{gather}

And from the definition of persistence lenght as the average sum of
projections of n bonds on the direction of the first bond when $n \to
\infty$ we have:

\begin{gather}
a = \sum_{i=1}^{n} \left<\mathbf{l}_{i}\right> \cdot (\mathbf{l}_{1}/l_{1})
\end{gather}
  
Then, when $n \to \infty$ the sum in the second term in the right hand
side of equation \ref{eq:char} can be rewritten as:

\begin{gather}
\sum_{j=1}^{n} \left<\mathbf{l}_{j}\right> \cdot l(\mathbf{l}_{1}/l_{1}) -(\mathbf{l}_{1}
\cdot \mathbf{l}_{1}) = al - l^{2}
\end{gather}  

In a similar fashion one can see that for the third term in
\ref{eq:char} for $n \to \infty$ the sum becomes $al-l^{2}$.
Then the expression for the second moment of the mean square
end-to-end distance when $n \to \infty$ will be given by:

\begin{gather}
\left<r^{2}\right> = nl^{2} + 2n(al-l^{2})\\
\left<r^{2}\right> = 2nla - nl^{2}
\end{gather}  

And the characteristic ratio for a chain where $n \to \infty$ is then:

\begin{eqnarray}
C_{\infty} & =  & \frac{\left<r^{2}\right>}{ nl^{2}}\\
          & =  & \frac{2nla - nl^{2}}{nl^{2}}\\
          & =  & \frac{2a}{l} - 1
\end{eqnarray}  

\section{Porod-Kratky or Worm Like Chain (WLC)}
If a polymer is modelled as a linear (not branched) continous and
homogenous chain its curvature at an arclength $s$ is given by the
tangential unit vector $\hat{\mathbf{t}}$\footnote{Note that we will
be using from now on the hat ($\hat{\mathbf{t}}$) notation to denote
unit vectors.}:

\begin{gather}
\hat{\mathbf{t}}(s)=\frac{\partial{\textbf{r}(s)}}{\partial{s}}
\end{gather}

here $\textbf{r}(s)$ is the position vector of a point in the chain
with respect to the coordinate origin.
The end-to-end vector is then defined as the integral sum of the chain
curvature over the contour length of the chain as:

\begin{gather}
\textbf{r}=\int_{0}^{L}\hat{\textbf{t}}(s)ds
\end{gather}

The average sum of the squared end-to-end distance is then:

\begin{eqnarray}
\label{eq:wlc}
\left<r^2\right> & = & \left<\textbf{r} \cdot \textbf{r}\right> \nonumber \\
               & = & \left<\int_{0}^{L}\hat{\textbf{t}}(s)ds \cdot \int_{0}^{L}\hat{\textbf{t}}(s')ds' \right>\nonumber \\
               & = & \int_{0}^{L} ds \int_{0}^{L} \left<\hat{\textbf{t}}(s) \cdot \hat{\textbf{t}}(s')\right>ds' \nonumber \\
               & = & \int_{0}^{L} ds \int_{0}^{L} \left<\cos{\theta}_{ss'}\right>~ ds' \nonumber \\
               & = & \int_{0}^{L} ds \int_{0}^{L} \exp^{-\frac{|s-s'|}{a}} ds' \nonumber \\
               & = & 2aL \{ 1 - \frac{a}{L}(1-\exp^{-\frac{L}{a}})\}
\end{eqnarray}

where  for   a  chain  of  infinite   length  $\left<\cos{\theta_{ss'}}\right>  =
\exp^{-\frac{|s-s'|}{a}}$, with $a$ being the persistence length.  The
last   equation  in   \ref{eq:wlc}   is  used   to   describe  a   so-called
worm-like-chain (WLC).

When $L \to \infty$ we see that $\left<r^{2}\right>=2aL$, and in such
case the characteristic ratio for a WLC becomes:

\begin{gather}
C_{\infty}=\frac{2aL}{nl^{2}}  
\end{gather}  

\section{Sequence Dependent}
Equation \ref{eq:e2e} can be expressed with respect to the reference
frame of the first bond, or bond-vector via transformation matrices as:

\begin{eqnarray}
\label{eq:real1}  
\mathbf{r} = \mathbf{l}_{1} & + & \mathbf{T}_{12}\mathbf{l}_{2}
+\mathbf{T}_{12}\mathbf{T}_{23}\mathbf{l}_{3} \nonumber \\
 & + & ... + \mathbf{T}_{12}\mathbf{T}_{23}...\mathbf{T}_{N-1,N} \mathbf{l}_{N}
\end{eqnarray}

where the transformation matrices $\mathbf{T}_{N-1,N}$ transform the
vectors $\mathbf{l}_{N+1}$ in coordinate frame $N+1$ to their
representation $\mathbf{l}_{N}$ in coordinate frame $N$.
Following Flory \cite{flory1969} equation \ref{eq:real1} can be
written as a matrix product in the following way:

\begin{gather}
\mathbf{r}= \begin{bmatrix}\mathbf{E}_{3}~ 0 \end{bmatrix} \mathbf{A}_{1:N}\begin{bmatrix} 0 \\ 1 \end{bmatrix}
\end{gather}

$\mathbf{A}_{1:N}$ is a serial product of generator matrices
$\mathbf{A}_{n}$ \cite{flory1969, maroun1988a, marky1994a},
$\mathbf{E}_{3}$ is the identity matrix of order 3, and $0$ is a
vector of necessary dimensions for matrix products to conform.

\begin{gather}
\mathbf{A}_{n} =
\begin{bmatrix}
\mathbf{T}_{n} & \mathbf{l}_{n} \\
0 & 1
\end{bmatrix}\\
\mathbf{A}_{1:N}=\mathbf{A}_{1}\mathbf{A}_{2}...\mathbf{A}_{N}
\end{gather}

The  generator  matrices  perform  a coordinate  transformation  which
incorporates the translation vectors $\mathbf{l}_{n}$ and the rotation
matrices $\mathbf{T}_{n}$.  It is within the product  of the generator
matrices  where  the  sequence  model  comes  into  play  since  every
generator  matrix contains  the specific-sequence  dependent geometric
information based  on the  rigid block model  for nucleic  acids.  For
complete  detail  of  the   transformation  of  step  parameters  into
euclidean  space, and how  the polymer  configuration space  is sampled
using the Gaussian sampling technique,  we refer the reader to Appendix
A of Luke Czapla's doctoral thesis \cite{czapla2009}.

Besides from being able to express the end-to-end distance using
generator matrices we can also find the persistence length as the
component in row three and column four of the generator matrix when
it's sampling has been done for a large number of conformations such
that the terms in the rotation matrix $\mathbf{T}_{n}$ approach very
small numbers.

\begin{gather}
\mathbf{P}_{N} = \left<\mathbf{A}_{1}\right> \left<\mathbf{A}_{2}\right> ... \left<
\mathbf{A}_{N-1}\right> \left< \mathbf{A}_{N}\right>\\
a = \lim_{N \to \infty} \left[\begin{array}{cccc}0 & 0 & 1 & 0\end{array}
    \right] \mathbf{P}_{N}
\left[\begin{array}{c} 0\\ 0\\ 0\\ 1\end{array}\right]
\end{gather}  

\section{Code Implementation}








%"length at which the orientation of the sequential bonds which make up
%a polymer chain, stop being correlated.  That is, if you have just two
%bonds, or a  few, they will be correlated, which is  the case in small
%molecules,  but, in  polymers, you  have  a long  chain of  sequential
%bonds. At some length, bonds  will become uncorrelated, but up to that
%length  they were  correlated, this  is what  is meant  by persistence
%length,  and,  in this  context  it's  obvious  that is  an  exclusive
%property of polymers." My understanding so far.





%\section{Suggested Reads}
%From Equilibrium Statistics of Plischke and Bergersen they suggest to
%read:
%Des Cloiseaoux and Janik ()
%Rubinstein and Colby (Polymer Physics)
\bibliography{biblio}
