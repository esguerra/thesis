\chapter{Persistence Length}
\label{appendix4a}
\bibliographystyle{nar}
Nucleic  Acids and  other  polymers can  be  understood as  mechanical
objects   \cite{marko2003,  nelson2004}   and   therefore  engineering
approaches can  be used for their understanding.   Usually the methods
followed by  the engineering approach  consider the polymer as  a long
continuous rod, and  are known as continuum elastic  theory. This type
of continuous  approaches leave little  space for taking  into account
the nature of  the subunits which make up the polymer,  that is, it is
mainly applicable  to homopolymers made  up of identical  subunits. In
nucleic acids this is not necesarily the case, a more general approach
should take into account  the possibility of having different subunits
making  up the  polymer.   Olson and  collaborators  have developed  a
sequence  dependent  model,  refered   to  as  the  "realistic"  model
\cite{olson1993}.  Such model is harmonic and depends on determination
of  force-constant analogs  derived from  X-Ray  cristallographic data
taken from  the Nucleic Acid Database  (NDB) \cite{go1976, olson1998}.
Within   the  context  of   the  "realistic"   model  Czapla   et  al.
\cite{czapla2006} have suggested a gaussian sampling methodology which
allows  the  determination  of  global  polymer  properties  like  the
persistence length $a$ following  a matrix approach suggested by Flory
\cite{flory1969} and expanded upon  by Olson et al. \cite{maroun1988a,
  marky1994a}

In what follows we summarize definitions of persistence length
and how it is computed using different models.
%The "realistic model" can be simplified to other  models, for example
%to the  freely rotating model, just by  making the  twisting constant
%zero, that is, the polymer will not be elastic when twisted.

\section{Persistence Length Definition}
%%Perhaps it would be good to say, "parallel and equivalent" perspect.
In  general there  are two  parallel and  analog perspectives  used to
define  the persistence  length  of  nucleic acids.   One  has a  more
abstract mathematical, or physical flavour, and it's understood as the
resistence  to  deformation  of;  a  curve in  space  (a  mathematical
object), or a thin rod (a physical object).  The other is a stochastic
one, where  it's understood as "a  measure of the  distance over which
the  direction of  the DNA  is maintained"  \cite{olson1995}.  In both
cases the persistence length definition  can be undestood as a measure
of polymer stiffness.

Within the context of the "physicists" definition we cite Marko's
and Nelson's definitions:

"Classical elasticity tells us that a thin, straight rod that is bent
into an arc has a bending energy $E=Bl/2R^2$, where $B$ is the bending
elastic constant of the rod, $l$ is the length of the rod and $R$ is
the radius of arc. Setting $R=l$ gives us the energy of a 1 radian
bend along the rod, and solving for when $E \kappa_{B}T$ gives us the
length of rod along which a thermally excited bend of 1 radian
typically occurs: $l B/\kappa_{B}T$. This is called the persistence
length..." John F. Marko and Simona Cocco, Physics World, March 2003

"In the elastic rod model of  a polymer, the elastic energy of a short
segment       of       rod       is       $dE=\frac{1}{2}       K_{B}T
[A\beta^2+Bu^2+C\omega^2+2Du\omega]  ds$.  Here $AK_{B}T$,  $CK_{B}T$,
$BK_{B}T$,  and $DK_{B}T$  are  the bend  stiffness, twist  stiffness,
stretch stiffness, and twist-stretch coupling, and ds is the length of
the segment.  (The quantities  A and  C are also  called the  bend and
twist  persistence  lengths.)"   Philip  Nelson,  Biological  Physics:
Energy, Information, Life, 2004.

From the "stochastic" perspective we cite Flory's definition:

"the average sum of the projections of all bonds $ j \geq i$  on bond
$i$ in an indefinetely long chain. The bond $i$ is taken to be remote
from either end of the chain, i.e., $1 \ll i \ll n$". Paul J. Flory,
Statistical Mechanics of Chain Molecules. 1969

This second perspective is more familiar to chemists, since it assumes
some type  of bonded connectivity  between polymeric units,  where the
bond can be either "real", or "virtual".

Using the persistence length $a$ one can classify biopolymers according to
their stiffness into three groups as shown in Table \ref{tab:pers}. 

\begin{table}[htbp]
\begin{center}  
\begin{tabular}{c|c|c}
\hline
Model Type      & Polymer Characteristic & $a$ to $L$ relation\\ \hline
Rigid Rod       & Rigid          &        $a \gg L$   \\
Gaussian chain  & Flexible       &        $a \ll L$   \\
Worm-like chain & Semi-flexible  &    $a \approx L$ \\
\hline
\end{tabular}
\label{tab:pers}
\end{center}
\end{table}











Notice that for $a \gg L$, there is a definition problem, since $L$
has to be large enough to be a good approximation to the definition of
persistence length, which is defined for an infinite chain length.

Worm-like-chain = Porod-Kratky = Freely Rotating Chain in limit l=0
and n=infinity






The flexibility of biopolymers can be roughly classified into three
groups;  rigid,  flexible,  and  semi-flexible.  The  models  used  to
understand biopolymers are usually good for say a rigid type, but fail
for a flexible polymer

by using the  ratio of persistence length to  the total contour length
of the polymer,


the following table illustrates such classification.



With this classification in mind it is




The validity of the models 





"length at which the orientation of the sequential bonds which make up
a polymer chain, stop being correlated.  That is, if you have just two
bonds, or a  few, they will be correlated, which is  the case in small
molecules,  but, in  polymers, you  have  a long  chain of  sequential
bonds. At some length, bonds  will become uncorrelated, but up to that
length  they were  correlated, this  is what  is meant  by persistence
length,  and,  in this  context  it's  obvious  that is  an  exclusive
property of polymers." My understanding so far.





Biopolymers can be either rigid or flexible. 

They can be classified  according to whether their persistence length ($a$)
is greater, smaller, or similar to the contour length ($L$) of the polymer.


\\

Rigid biopolymers:
actin, microtubules
\\

Flexible biopolymers:
\\

Semi-flexible biopolymers:
High force extension DNA.
\\

If the persistence length is of the same order of the length of the
polymer, then the polymer is classified as  semi-flexible

\begin{table}[htbp]
\begin{center}
\begin{threeparttable}
\begin{tabular}{l|r|c}
\hline
Polymer               & $a$ (nm) & Citation  \\ \hline
Polyglicine           & 0.6      &   Flory \tnote{a} \\
Poly-L-alanine        & 2        &   Flory \tnote{a}  \\
Poly-L-proline        & 22       &   Cantor and Schimel \\
B-DNA                 & 51       &            \\
A-RNA                 & 70-80    &   Hagerman \\
$\alpha$-helix        & 80-100   &   Kaushik  \\
coiled-coil           & 150-300  &   Kaushik  \\
Neurofilament         & 500      &   Nelson \cite{nelson2004}   \\
Intermediate Filament & 1000     &   Kaushik  \\
F-Actin               & 17000    &   Kaushik  \\
Microtubule           & 5200000  &   Kaushik  \\
\hline
\end{tabular}
\begin{tablenotes}
\item [a] Calculated using  characteristic ratios reported in Table 3
  of Flory's \cite{flory1969} with C$^{\alpha}_{i}$ to C$^{\alpha}_{i+1}$ virtual
  bond length $\nu = 3.80$ \AA.
\end{tablenotes}
\caption{Persistence lengths for some biopolymers with filament structures.}
\end{threeparttable}
\end{center}
\end{table}


\section{end-to-end}
The end-to-end vector  $r$ is the vector which connects  the ends of a 
polymer chain.  It  can be defined  as the sum  of the vectors connecting the
monomer units in a chain. These connecting vectors are sometimes called
virtual  bond vectors $l$.  

From  the end-to-end  vector  the quantity
which is usually of interest is it's magnitude.

\begin{gather}
\label{eq:end2end}
r = \sum_{i=1}^{n} l_{i}\\
r^2 = r \cdot r = \sum_{i,j}l_{i} \cdot l_{j}
\end{gather}
Equation \ref{eq:end2end}, can also be written:
\begin{gather}
r^2 = \sum_{i}l_{i}^{2} + 2 \sum_{i\neq j} l_{i} \cdot l_{j}
\end{gather}  

To  describe a  polymer  it's  necessary to  think  about the  various
conformations it can adopt  due to its flexibility, therefore, it
is important  to think of polymer  related quantities in  terms of the
average of their possible conformations. For the end-to-end vector the
average of  its values  is denoted  as $<r>$, and  the average  of its
norm, also called the second moment of the end-to-end distribution, is
denoted by $<r^2>$:
\begin{gather}
\label{eq:secmom}  
<r^2>=\sum_{i}<l_{i}^2> + 2\sum_{i<j}<l_{i} \cdot l_{j}>
\end{gather}  

When there is no correlation between succesive bonds we can write:

\begin{gather}
\label{eq:nocorr}
<l_{i} \cdot l_{j}> = 0
\end{gather}

So that equation \ref{eq:secmom} keeps only the bond auto-correlation term:

\begin{gather}
<r^2> = \sum_{i}<l_{i}^2> = n<l^2>
\end{gather}  

This equation is used to describe a so-called freely-jointed chain.





\section{Models}

Nelson in book says:



\subsection{Kuhn - Freely Jointed Chain (FJC)}

\subsection{Porod-Kratky - Worm Like Chain (WLC)}

\subsection{Olson - Realistic}

The Hamiltonian for a \cite{czapla2009}





\section{Suggested Reads}

From Equilibrium Statistics of Plischke and Bergersen they suggest to
read:
Des Cloiseaoux and Janik ()
Rubinstein and Colby (Polymer Physics)
\bibliography{biblio}
