\chapter{Clustering Analysis (CA)}
\label{appendix_a}
\bibliographystyle{nar}

\section{General Methodology}
We considered  each of the 20  structures as a vector  composed of the
six  base step  parameters.  We  grouped these  vectors  using cluster
analysis following an automated process shown to succesfully reproduce
well  known patterns  of the  periodic table  from a  selected  set of
variables, such as, electronegativity, ionization potential, and other
elemental properties \cite{restrepo2004}.  The procedure followed here
is  an adaptation  of the  clustering used  to construct  the periodic
table.

We start by normalizing the vectors of step parameters,
\begin{gather}
\label{eq:normalization}  
\overline{x}_{jA}=\frac{x_{jA}-x_{jmin}}{x_{jmax}-x_{jmin}}
\end{gather}
where ${x}_{jA}$ is the value  of the step parameter \textit{j} of the
structure A and $x_{jmin}$ and  $x_{jmax}$ are the minimum and maximum
values for a particular step parameter \textit{j} \cite{restrepo2006}.
Then,  using  the  software  package \textsf{R}  \cite{ihaka1996},  we
cluster these vectors into groups.  These groups can be displayed in a
tree  representation,  also called  a  dendrogram,  or  in biology,  a
phylogenetic tree (see Figure~\ref{fig:tree}).

To cluster  these vectors into groups,  it's necessary  to define the
distance between  the vectors. In  this work we used  three distance
definitions.   These distances  are  often referred  to as  Manhattan,
Euclidean  and   maximum  distances.  The  first   two  distances  are
particular cases of what is known as Minkowski's metric
\begin{gather}
d(X,Y)= \Big( \sum_{i=1}^N |x_i-y_i|^k \Big)^\frac{1}{k}
\end{gather}
where $d(X,Y)$ refers to the distance between two vectors $X$ and $Y$,
$N$  is  the  dimensionality of  the  vector,  for  the case  of  step
parameters, $N$ is  six.  In the case where \textit{k}  is equal to 1,
the  definition  corresponds to  the  Manhattan  distance (a  distance
measured by  following along the edges  of blocks). In  the case where
\textit{k} is equal to 2, we have the familiar Euclidean distance. The
remaining distance, that is, the maximum distance, is defined by:
\begin{gather}
d(X,Y) = max |x_{i}-y_{i}|
\end{gather}
where  the  distance  between  vectors  $X$ and  $Y$  is  the  maximum
difference between vector variables.

With  these distance  definitions,  we use  a hierarchical  clustering
method.

The clustering algorithm first finds the two closest vectors (given by
one of  the distance  definitions) and groups  them together.  Then it
compares the  distance of the elements  in the newly  formed group and
the  elements remaining  to be  grouped, according  to  the particular
clustering method.  For example,  the single linkage clustering method
takes  the  minimum  distance   between  elements  as  the  clustering
criterion.   Such  an  approach  would  (as  all  other  agglomerative
hierarchical methods do), group together the closest vectors given the
distance definition, and then would use the method definition (minimum
distance) to compare the distance of the elements of the group, to the
elements  which  remain  ungrouped,   or  to  the  elements  of  other
groups. As new  groups are formed the process  is repeated following a
hierarchical order,  that is, whatever  distance is smaller  gives the
grouping  criterion.   We   have  used  four  hierarchical  clustering
methods, the description of these methods follows in the next section,
"Hierarchical Methods".

For  every  possible combination  of  clustering  method and  distance
definition we  obtain a dendrogram. The combination  of three distance
definitions and four clustering  methods leads to 12 clustering trees.
These trees are not all exactly  the same but show recurring groups of
conformers.  To find the groups  which are repeated among the trees, a
consensus  analysis  is  performed  using  the  \textsf{clue}  package
\cite{hornik2005} implemented  in \textsf{R}. The  resulting consensus
tree is illustrated in Figure~\ref{fig:eucl_cons}.

\section{Hierarchical methods}
The hierarchical clustering methods used were:

\begin{enumerate}
\item{ \textit{Single linkage  clustering}, where the minimum distance
  between elements of each cluster is taken as clustering criteria.
\begin{gather}
D(X, Y)=min\{d(x_i, y_j): x_i \in X, y_j \in Y \}
\end{gather}
where  $X$ and  $Y$ are  vectors, and  $d(x_i, y_j)$  is  the distance
between cluster elements.  }

\item{  \textit{Complete   linkage  clustering},  where   the  maximum
  distance between cluster elements is the clustering criteria.
\begin{gather}
D(X, Y)=max\{d(x_i, y_j): x_i \in X, y_j \in Y \}
\end{gather} }

\item{ \textit{Average linkage  clustering}, the mean distance between
  elements of each cluster is taken as clustering criteria.
\begin{gather}
D(X, Y)=\frac{1}{N_x  * N_y} \sum_{i=1}^{N_x}  \sum_{j=1}^{N_y} d(x_i,
y_j)
\end{gather}
where  $N_x$  and $N_y$  are  the  number  of elements  in  respective
clusters.  }

\item{ \textit{Centroid linkage clustering}, uses the distance between
  cluster centroids, as clustering criteria.
\begin{gather}
D(X, Y)=d(\overline{x}, \overline{y})\\
\overline{x} = \frac{1}{N_x} \sum_{i=1}^{N_x} x_i\\
\overline{y} = \frac{1}{N_y} \sum_{i=1}^{N_y} y_i
\end{gather} }

\item{ \textit{Ward's  Method}, uses the  error sum of  squares (ESS).
%This Error Sum of Squares might  be the same as the residual sum of
%squares which is important in regression models.
\begin{gather}
D(X,Y)=ESS(XY) -[ESS(X) + ESS(Y)]\\
ESS(X)=  \sum_{i=1}^{N_x} \left|
x_i -\frac{1}{N_x}\sum_{j=1}^{N_x} x_j\right|^2
\end{gather} }
\end{enumerate}

As an example lets think of a case where we have five structures. Each
one of  them is descibed by  a bidimensional vector  as illustrated in
Table~\ref{tab:data}.
\begin{table}
\centering
\begin{tabular}[h]{|c|c|c|}
\hline
Structure & Property I & Property II\\
\hline\hline
1  &     1.00  &  5.00 \\
\hline
2  &    -2.00  & 6.00 \\
\hline
3  &      2.00  & -2.00 \\
\hline
4  &     -2.00  & -3.00 \\
\hline
5  &     3.00  &  -4.00 \\
\hline
\end{tabular}
\caption{Example of  structures, considered as  bidimensional vectors,
  to be clustered  using the average linkage method  and the Manhattan
  distance.}
\label{tab:data}
\end{table}

The  first step  is  to chose  a  distance definition.   We chose  the
Manhattan distance.  The Manhattan distance  values between structures
can  be   displayed  in   a  lower  triangular   matrix  as   seen  in
equation~\ref{eq:man}
\begin{gather} 
d(X, Y)=
\begin{vmatrix}
   & 1  &  2   & 3 & 4 & 5 \\
1  & 0  &      &   &   &   \\
2  & 4  &  0   &   &   &   \\
3  & 8  & 12   & 0 &   &   \\
4  & 11 &  9   & 5 & 0 &   \\
5  & 11 & 15   & 3 & 6 & 0 \\
\end{vmatrix}
\label{eq:man}
\end{gather}

Let's calculate explicitily  the Manhattan distance between structures
2 and 3,

\begin{gather}
d(2, 3)= |-2.00 - 6.00| + |2.00 - -2.00| = 12
\end{gather}

Now that we have calculated the distances we need a clustering method,
in this case, we will use the average linkage clustering method. There
are two hierarchical techiques called agglomerative, or bottom-up, and
divisive, or  top-down. We will use the  agglomerative technique, that
is, going  from the bottom where  no objects are grouped,  to the top,
where all objects  constitute one final group. The  first step is then
to group whatever  structures are closer, that is,  structures 3 and 5
($d(3, 5)=3$). Now  we find the mean distance  between the elements of
this  cluster  and  the  remaining unclustered  structures,  that  is,
structures 1, 2 and 4, we obtain the following mean distances
\begin{gather}
D(\{3,5\}, 1)=\frac{1}{2*1}*(8+11) = 4.5 \label{eq:dist}\\
D(\{3,5\}, 2)=\frac{1}{2*1}*(12+15) = 13.5\\
D(\{3,5\}, 4)=\frac{1}{2*1}*(5+6) = 5.5
\end{gather}
Since  the distances between  \{3, 5\}  and all  remaining unclustered
vectors is  higher than  the distance between  vectors 1 and  2 ($d(1,
2)=4$) then \{1, 2\} are grouped. The following value, in hierarchical
increasing   order   is   4.5    between   \{3,   5\}   and   1   (see
equation~\ref{eq:dist}),  but since  1 and  2 are  already  grouped we
can't group  \{3, 5\} with 1.  The next value, following  the lower to
higher hierarchy,  is 5 ($d(3, 4)=5$),  but we have  already grouped 3
with 5, so we have to  keep advancing in the hierarchy. The next value
is 5.5, which  corresponds to grouping \{3, 5\} with  4, so we cluster
them. The only  remaining possibility for grouping is,  group \{1, 2\}
and \{4, 3, 5\}, so we do it as illustrated in Figure~\ref{fig:tree}.
\begin{figure}[t]
\centering
\includegraphics[scale=0.4]{Appendix/appendixtree.png}
\caption{Clustering  tree  for   5  bidimensional  vectors  using  the
  Manhattan  distance definition  and the  average  linkage clustering
  method.}
\label{fig:tree}
\end{figure}


%\newline
%Distance can be defined by Minkowski's metric:
%\begin{gather}
%d(X,Y)= \Big( \sum_{i=1}^N |x_i-y_i|^k \Big)^\frac{1}{k}
%\end{gather}
%In the particular case when $k=1$ we have the definition of the
%Manhattan or taxi-cab distance and when $k=2$ it's the familiar
%Euclidean distance.

%Finally, we can also define a maximum distance as the maximum
%difference between vectors variables.
%\begin{gather}
%d(X,Y) = max |x_{i}-y_{i}|
%\end{gather}

\bibliography{biblio}


