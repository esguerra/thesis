\chapter{Standard reference frame and local parameters}
\label{appendix_1a}
\bibliographystyle{nar}  In   addition  to  the   description  of  RNA
structures  at the  level of  torsion  angles, one  can also  describe
structure  in  terms  of  the  spatial  arrangements  of  adjacent  or
associated bases.  The structural description  of RNA used  here comes
from the program \textsf{3DNA} \cite{lu2003}, which reports three sets
of parameters that define the local arrangements of bases.
\begin{enumerate}
\item Base-pair parameters,
\item Base (base-pair) step parameters,
\item Base (base-pair) local helical parameters.
\end{enumerate}
The base or  base-pair parameters are the quantities  that bring into
coincidence  coordinate  frames  on   two  objects  using  ideas  from
classical mechanics.
%% The first two are based on cartesian coordinates
%% (``objectcentered'', reference frame),
%% whereas the third is referred as helical coordinates
%% (non-object centered)
The first two sets of parameters are based on Cartesian coordinates of
two bases or  base pairs, whereas the third set  of helical parameters,
resembles cylindrical coordinates and are based on the single rotation
that  brings  coordinate  frames  on  two  bases  or  base-pairs  into
coincidence (Chasles's theorem) \cite{babcock1994}.

\section{Base-pair and base-step parameters}
In \textsf{3DNA} one  starts with a Protein Data  Bank (PDB) formatted
\cite{berman2000}  file,  which   is  usually  based  on  experimental
information\footnote{A PDB file is sometimes the result of theoretical
  modeling.}   and  which can  be  downloaded  from  the Nucleic  Acid
Database (NDB)  or PDB.  This file contains  the Cartesian coordinates
and other  information, such as sequence composition,  about the given
structure of  the atoms.  With  this experimental data one  performs a
least-squares fit to a  standard reference frame \cite{olson2001}. The
\textsf{octave}                        script                       at
\url{http://www.eden.rutgers.edu/~esguerra/RNA/scripts.html}   provides
a useful tutorial example.  The coordinate origin which is embedded in
the standard  reference frame, is  used for the determination  of both
base and base-pair parameters.  In  the case of single unpaired bases,
the  program keeps the  origin of  one base  of an  ideal Watson-Crick
pair.    The   definition   of    this   frame   is   illustrated   in
Figure~\ref{fig:standard}

\begin{figure}[htbp]
\centering
\includegraphics[scale=0.8]{Chapter1/standard.png}
\caption{Standard    reference    frame    of   an    A-T    base-pair
  \cite{olson2001}.  The \textit{y}-axis (dashed green line) is chosen
  to be  parallel to the line  connecting the C1$^{'}$  of adenine and
  the  C1$^{'}$  of  thymine   associated  in  an  ideal  Watson-Crick
  base-pair. The \textit{x}-axis is  the perpendicular bisector of the
  C1$^{'}$  -  C1$^{'}$  line,  and  the  origin  is  located  at  the
  intersection of  the \textit{x}-axis and the line  connecting the C8
  atom of adenine  and the C6 atom of  thymine. The \textit{z}-axis is
  normal  to the  base-pair plane  (defined in  a positive  sense with
  respect to the leading base, here A) and the direction of the x-axis
  is defined by the cross  product of the $\hat{x}$ and $\hat{y}$ unit
  vectors.}
\label{fig:standard}
\end{figure}

After the  coordinate origins for  two consecutive base  of base-pairs
comprising a  step have  been computed, then  a so-called  middle step
triad (MST)  \cite{lu1997} is defined. Defining the  middle step triad
is described by the following procedure:

1) Find the angle $\Gamma$ between consecutive normals, \textit{i.e.},
\textit{z}-axes. Since these are unit vectors, the angle is defined by
the scalar product:

\begin{gather}
\Gamma = \cos^{-1} (\hat{z}_i \cdot \hat{z}_{i+1})
\end{gather}

2)  Then find the  vector which  is perpendicular  to the  two normals
(\textit{z}-axis). This  vector is obtained from the  cross product of
the  consecutive \textit{z}-axes  (that is,  the normal  to  the plane
formed by the two vectors). This axis is called the roll-tilt axis and
is normalized to form the unit vector $\hat{r_t}$,

\begin{gather}
\hat{r_t} = \frac{\hat{z}_i \times \hat{z}_{i+1}}{|\hat{z}_i \times
\hat{z}_{i+1}|}
\end{gather}

%% Note: Why do you need a matrix? Isn't this always in 3D and therefore
%% just vectors would do? Why the more general matricial way instead of
%% the vectorial representation?
3) To make consecutive \textit{z}  vectors coincide, one uses a linear
homogeneous transformation  $R(\theta)$ about the  roll-tilt axis such
that the  original orientation matrices $T_i$ and  $T_{i+1}$, i.e. the
set  of unit  vectors  along  the x,  y  and z-axes  of  each base  or
base-pair specified in the columns of  each, are rotated by $ \theta =
\pm \Gamma /  2$ to yield the transformed  $T_i^{'}$ and $T_{i+1}^{'}$
orientation matrices.
%% NEED to include GRAPHICS of ANGLES and COORDINATES in general.

\begin{gather}
T_i^{'} = R_{rt}(\pm \Gamma/2) T_{i} \\
T_{i+1}^{'} = R_{rt}(\mp \Gamma/2) T_{i+1}
\end{gather}

The origin  for the middle step  triad (MST) is the average  of the position
vectors $r_{i}$ and $r_{i+1}$ for the $i$ and $i+1$ reference frames,

\begin{gather}
r_{MST} = \frac{r_i + r_{i+1}} {2}
\end{gather}

4) Again  using the  dot product  one can find  the angle  between the
transformed  $\hat{y}^{'}$  vectors.   This  angle  is  equal  to  the
magnitude of the Twist ($\Omega$)  in the base of consecutive bases or
base-pairs.  The  dot product  of the $z$  unit vector for  the middle
step triad  (MST) $\hat{z}_{MST}$ with  the vector resulting  from the
cross product  of $\hat{y}_{i}^{'}$ and  $\hat{y}_{i+1}^{'}$ gives the
sign  of  $\Omega$.  Since   the  transformed  \textit{x-y}  plane  is
orthogonal  to $\hat{z}$  then this  applies  in the  same manner  for
\textit{x},
%%The vector  resulting from  (yi X  yi+1 has  got  to beparallel or
%%anti-parallel to zMST, there's no other possibility.

\begin{gather}
\Omega = cos^{-1}(\hat{y}_{i}^{'} \cdot \hat{y}_{i+1}^{'})\\
(\hat{y}_{i}^{'} \times \hat{y}_{i+1}^{'}) \cdot \hat{z}_{MST} > 0, \quad \textrm{then} \ \Omega > 0\\
(\hat{y}_{i}^{'} \times \hat{y}_{i+1}^{'}) \cdot \hat{z}_{MST} < 0, \quad \textrm{then} \ \Omega < 0
\end{gather}

%%if normalized beforehand then the rule would be,
%%\begin{gather}
%%(\hat{y}_{i}^{'} \times \hat{y}_{i+1}^{'}) \cdot \hat{z}_{mst} = 1, \quad %%then \ \Omega > 1\\
%%(\hat{y}_{i}^{'} \times \hat{y}_{i+1}^{'}) \cdot \hat{z}_{mst} =
%%-1,\quad then \ \Omega < -1
%%\end{gather}

5) With more scalar products one  can find other angles, such as the phase
angle $\phi$,
%% Show in figure.

\begin{gather}
\phi = cos^{-1}(\hat{rt} \cdot \hat{y}_{MST})\\
(\hat{rt} \times \hat{y}_{MST}) \cdot \hat{z}_{MST} > 0, \quad \textrm{then} \ 180 \geq \phi \geq 0\\
(\hat{rt} \times \hat{y}_{MST}) \cdot \hat{z}_{MST} < 0, \quad \textrm{then} \ -180 \leq \phi \leq 0
\end{gather}

6) The  roll $\rho$  and tilt $\tau$  angles, which are  the remaining
angular degrees of  freedom for step parameters, are  defined in terms
of the bending angle and the phase angle:
%as if we were doing a change of variables from cartesian to polar.

\begin{gather}
\rho = \Gamma cos (\phi)\\
\tau = \Gamma sin (\phi)
\end{gather}

To  get the remaining  three translational degrees of  freedom for
step  parameters  ($Dx,  Dy,  Dz$)  one  needs  to  express  the
displacement vector in the middle step triad frame:

\begin{gather}
[D_xD_yD_z]=T_{MST}(r_{i+1} - r_{i})
\end{gather}

The  procedure  to  compute  the base-pair  parameters  is  completely
analogous.   The  opening  $\omega$,  buckle $\kappa$,  and  propeller
$\sigma$  are the  analogs of  twist $\Omega$,  roll $\rho$,  and tilt
$\tau$, and the middle step triad is called the middle base triad MBT.
The axes  which are made to  coincide are the  \textit{y}-axes and not
the \textit{z}-axes as in the base-pair step case \cite{lu1997}.

The parameters obtained by  this procedure are depicted graphically in
Figure~\ref{fig:allparam}.
\begin{figure}[htbp]
\centering
\includegraphics[scale=0.6]{Chapter1/allparam2.png}
\caption{Illustration   of   base-pair   and  base   step   parameters
  \cite{lu2003}.   As seen  in the  upper  right corner  the base  and
  base-pair step parameters correspond  to the three translational and
  three rotational degrees or freedom which describe the geometry of a
  rigid-body. Thus the three  translational degrees of freedom, Shift,
  Slide, and Rise, are expressed  as linear displacements along the x,
  y, and z axis,  and the three  rotational degrees of  freedom, Tilt,
  Roll, and Twist, as angular displacements around x, y, and z.}
\label{fig:allparam}
\end{figure}


%\begin{gather}
%\gamma = \cos^{-1} (\hat{y}_{iII} \cdot \hat{y}_{iI})\\
%bo = \hat{y}_{iII} \times \hat{y}_{iI}\\
%\hat{bo} = \frac {bo}{\sqrt{bo \cdot bo}}\\
%T_{iII}^{'} = R_{bo}(+\frac{\gamma}{2}) T_{iII} \\
%_{iI}^{'} = R_{bo}(-\frac{\gamma}{2}) T_{iI}\\
%r_{mbt} = \frac{(r_{iII} + r_{iI})} {2}\\
%\omega = cos^{-1}(\hat{x}_{iII}^{'} \cdot \hat{x}_{iI}^{'})\\
%(\hat{x}_{iII}^{'} \times \hat{x}_{iI}^{'}) \cdot \hat{y}_{MBT} > 0, \quad then \ \omega > 0\\
%(\hat{x}_{iII}^{'} \times \hat{x}_{iI}^{'}) \cdot \hat{y}_{MBT} < 0, \quad then \ \omega < 0\\
%\phi^{'} = cos^{-1}(\hat{bo} \cdot \hat{x}_{MBT})\\
%(\hat{bo} \times \hat{x}_{MBT}) \cdot \hat{y}_{MBT} > 0, \quad then \ 180 \geq \phi^{'} \geq 0\\
%(\hat{bo} \times \hat{x}_{MBT}) \cdot \hat{y}_{MBT} < 0, \quad then \ -180 \leq \phi^{'} \leq 0\\
%\kappa = \gamma cos (\phi^{'})\\
%\sigma = \gamma sin (\phi^{'})\\
%[S_xS_yS_z]=T_{MBT}(r_{iI} - r_{iII})
%\end{gather}

\section{Local helical parameters}

Local helical parameters are determined using Chasles's theorem, which
states \cite{babcock1994}:
\begin{quote}
``\textit{One can  always transport  a free rigid  body from one  position and
  orientation  to  another  position   and  orientation  by  a  single
  continuous motion along a unique axis of rotation.}''
\end{quote}

\noindent For  the case of  three-dimensional nucleic acid  base steps
what  this means  is  that, instead  of  rotating around  a series  of
reference-frame  centered  axes  and  then translating  along  another
reference-frame centered  axis, one rotates about  and also translates
along   only   one  common   axis,   which   is  not   reference-frame
centered. This allows us to  define the orientation of a local helical
axis (or unique rotational-translational  axis) as a unit vector given
by equation~\ref{eq:helaxis}:

\begin{gather}
\label{eq:helaxis}  
h=\left[ \begin{array}{c}
h_x\\
h_y\\
h_z
\end{array} \right]
\end{gather}
where:
\begin{gather}
h_x = \frac{\tau}{\Omega_h}, \qquad h_y = \frac{\rho}{\Omega_h},
\qquad h_z = \frac{\Omega}{\Omega_h}
\end{gather}

\begin{gather}
\Omega_h = \sqrt{\tau^2 + \rho^2 + \Omega^2}
\end{gather}

The local helical axis can be defined alternatively \cite{bansal1995}
as a cross product:

\begin{gather}
h = (x_2 - x_1) \times (y_2 - y_1)
\end{gather}
where the  $x$ and $y$ refer to  the reference frames on  base pairs 1
and 2.


\bibliography{biblio}


