\chapter{Dimension Reduction}
\label{pca}
\bibliographystyle{nar}
\section{Principal Component Analysis}
Given  a set  of data  it's important  to check  before analysis  if the
dimensions of such  set can be reduced. A very  common method to check
this is called Principal Component Analysis (PCA).

The PCA  method can be defined  in a strictly mathematical way as the
method which; finds the Principal  Components (PCs) of  a dataset by
doing  "an orthogonal  linear  transformation of  a  set of  variables
optimizing certain algebraic criterion" \cite{jolliffe2002}.

\begin{gather}
  \mathbf{y} = \mathbf{T} \mathbf{x}
\end{gather}

Where $\mathbf{T}$ is an orthogonal linear transformation matrix of
dimension $k$ by $n$, $\mathbf{x}$ is the "original" data matrix of
dimension $n$ by $m$, and by definition of the matrix product
$\mathbf{y}$ is a matrix of dimension $k$ by $m$.

From the linear tranformation expression it's clear that if $k=m$ then
the transformation matrix is just a rotation matrix, and in the case
where $k<m$ then the transformation matrix is also reducing the
dimension of the "original" data.

One common algorithm to find such transformation ($\mathbf{T}$) is the
following: 

\begin{enumerate}
\item{Substract the mean from the data matrix $\mathbf{x}$.
\begin{equation}
\mathbf{\Omega} = \mathbf{x} - \mathbf{\bar{x}_{m}}
\end{equation}
}
\item{Find the covariance matrix for $\mathbf{\Omega}$.
\begin{equation}
\mathbf{\Sigma} = \frac{\mathbf{\Omega}^{-1} \mathbf{\Omega}}{(1-n)}
\end{equation}  
}  
\item{Diagonalize the covariance matrix $\mathbf{\Sigma}$.
\begin{equation}
\mathbf{T}^{\mathrm{T}} \mathbf{\Sigma} \mathbf{T} = \mathbf{\Lambda}
\end{equation}  
}  
\item{Organize $\mathbf{T}$ from the highest to lowest eigenvalues in
$\mathbf{Lambda}$.}
\end{enumerate}  

Obtaining the eigenvalues and eigenvectors of $\mathbf{\Sigma}$ means
that we have found our transformation matrix $\mathbf{T}$, which can
be used to either rotate the original data space to an orthonormal
one, or to reduce the dimensionality of the data space by choosing
$k<m$, depending on the weight of the eigenvectors in
$\mathbf{\Lambda}$. The $k$ rows of $\mathbf{y}$ are named Principal
Components.

There is a large amount of bibliography which refers to the statistics
of Principal Components Analysis, and to sofistications which are not
included in this brief appendix. An interested reader can find great
help in the following web addresses:

\url{http://www-stat.wharton.upenn.edu/~buja/script-Buja-CU-2009-06-pca.R}
\url{http://www.snl.salk.edu/~shlens/pca.pdf}
\url{http://www.cse.unr.edu/~bebis/MathMethods/PCA/lecture.pdf}
\bibliography{biblio}


