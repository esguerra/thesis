\chapter{RNA Base-Pairs}
\label{basepairs} 
\bibliographystyle{nar}
\section{Canonical and Noncanonical Base-pairs}
As  shown   in  Figure  \ref{fig:saenger28},  there   can  be  various
base-pairing patterns between heterocyclic  bases in nucleic acids due
to the  variety of possible  hydrogen bonding interactions.   The most
prevalent hydrogen  bonding pattern  in nucleic acids  is that  of the
canonical  Watson-Crick  (W-C)  base-pair.   Other  possible  patterns
besides the W-C one are known as non-canonical base-pairs and are more
common in  RNA than in DNA.   We used the  3DNA \cite{lu2003} software
package  to find  all base-pairs  in  a non-redundant  dataset of  RNA
structures  obtained by X-ray  crystallography with  resolution better
than 3.5  \AA~ downloaded from the  protein data bank  (PDB).  We also
constrained our search to helical regions, which are defined as having
three  consecutive base-pairs  or more  which need  not  be covalently
bonded by the  sugar-phosphate backbone between consecutive base-pairs
\cite{olson2009}.

Our dataset is  non-redundant in the sense that  from the main source
of RNA structural information, which is the ribosome, we used only one
of the  available structures  per organism, that  is, one for  each of
\textit{Deinococus   radiodurans},   \textit{Haloarcula  marismortui},
\textit{Escherichi          coli},         and         \textit{Thermus
  thermophilus}. Table~\ref{tab:dbase}  shows in detail  the number of
bases per  RNA type in  our dataset. It's  interesting to see  that in
general the content of  G and C, is higher than that  of A and U which
might be due to the  highest overall stability of G$\cdot$C base-pairs
against A$\cdot$U base-pairs.
\begin{table}[H]
\begin{center}
\begin{tabular}{|l|c|r|r|r|r|}
\hline
RNA Type & \multicolumn{1}{p{2cm}|}{Number of Structures} & \multicolumn{1}{c|}{G} &
\multicolumn{1}{c|}{C} & \multicolumn{1}{c|}{A} &
\multicolumn{1}{c|}{U} \\ \hline \hline
small helices & 78 & 891 & 753 & 404 & 442 \\ \hline
drug-RNA & 36 & 932 & 862 & 365 & 433 \\ \hline
protein-RNA & 207 & 4001 & 3457 & 1771 & 1731 \\ \hline
protein-tRNA & 9 & 175 & 155 & 98 & 87 \\ \hline
rRNA & 13 & 3866 & 2949 & 1939 & 1785 \\ \hline
tRNA & 13 & 205 & 159 & 124 & 112 \\ \hline
ribozyme & 113 & 2434 & 2086 & 1438 & 1150 \\ \hline
Total & 469 & \multicolumn{1}{c|}{12504} & \multicolumn{1}{c|}{10421} & \multicolumn{1}{c|}{6139} & \multicolumn{1}{c|}{5740} \\ \hline
\end{tabular}
\caption{Classification of RNA Types in Non-Redundant Dataset at less
  than 3.5 \AA~(For Base-Pairs in Helices of 3 base-pairs or more).}
\label{tab:dbase}
\end{center}
\end{table}

In Table  \ref{tab:bpcomp} we show  all possible base-pairs  formed by
unmodified nucleotides in our dataset,  it's clear from the table that
the G$\cdot$C  and A$\cdot$U base-pairs  dominate widely the  space of
possible  RNA base-pairs  in helical  regions  making up  80\% of  all
base-pairs  and  if  we  only  count those  that  form  canonical  W-C
base-pairs, they make up about 73\% of all base-pair in helical regions.

As will be show later on, a considerable amount of A$\cdot$U's pair in
the Hoogsteen  fashion. There are a few  examples of U$\cdot$A$\cdot$U
triplets made by W-C and  Hoogsteen base-pairs in RNA helical regions,
which are the focus of our dataset.

The third base in such associations almost always comes from
a  single-stranded  region.  The  ``melted''  Watson-Crick  pairs  are
structurally perturbed compared to the canonical pairs. The bases tend
to be  more widely spaced in  terms the so-called Stretch  (Fig. 1) or
slipped  past one another  via Shear,  with concomitant  disruption of
hydrogen-bonding.  About 20\% of  the helical base pairs associate via
non-canonical forms.






3.2. Dominant base pairs

We classified the RNA base-pairs  in our dataset using three criteria.
(1) The Leontis-Westhof edge classification scheme \cite{leontis1998}.
(2)  The rotational  and translational  base-pairing  rigid-body
parameters,  Shear, Stretch, Stagger,  Buckle, Propeller  and Opening.
(3) The location of base-pairs  in helices, that is, their location in
``intact''  covalently bonded  sugar-phosphate backbones,  and  in the
ends   of   ``quasi-continuous''    helices   with   breaks   in   the
sugar-phosphate backbone.

Most  ($\sim$90\%)  of the  base  pairs  in  the RNA  helical  domains
associate  in   one  of  seven   distinct  hydrogen-bonding  patterns:
canonical Watson-Crick G$\cdot$C and A$\cdot$U pairs; wobble G$\cdot$U
pairs;  sheared  G$\cdot$A pairs;  Hoogsteen  A$\cdot$U pairs,  wobble
U$\cdot$U pairs;  and Watson- Crick-like  G$\cdot$A pairs (Fig.  3 and
Table 2). The relative populations of the canonical and the two wobble
complexes are roughly comparable to those found previously from visual
characterization  of   rRNA  base  pairs  [36].   The  constraints  of
double-helical structure and the consideration of other RNA structures
contribute to the lower relative abundance of the other pairs.

3.3. Structural context

Each of the dominant base  pairs has a unique spatial signature within
the RNA helical domains (Table 3). That is, some base pairs occur with
greater frequency within the interior  than at a ``nick'' or at either
end  of  a  stacked  double-helical  fragment.  Other  pairs  resemble
intercalating ligands  in that  they insert between  covalently linked
base pairs or stack  at the end of a duplex. Like  a nick in a regular
DNA duplex,  a ``nick'' in the  RNA helices corresponds to  a break in
the  covalent  link between  stacked  base  pairs.  The RNA  backbone,
however, is  ``broken'' in the sense  that the stacked  bases occur in
sequentially distant nucleotides.   Thus, the A$\cdot$U Hoogsteen pair
(A$\cdot$U$_{H}$) stands  out from  other base pairs  in terms  of its
intercalative   properties  and   unlikely  presence   within  intact,
chemically continuous  dinucleotide steps. By  contrast, the G$\cdot$A
Watson-Crick-like pair (G$\cdot$A$_{WC}$) has a higher propensity than
other base  pairs to  flank ``nicks'' within  quasi-continuous helical
domains, and  the sheared G$\cdot$A  pair (G$\cdot$A$_{s}$) to  lie at
the  ends   of  stacked   helical  arrays.  The   canonical  G$\cdot$C
Watson-Crick  pair  (G$\cdot$C$_{WC}$)   differs  from  its  A$\cdot$U
counterpart  (A$\cdot$U$_{WC}$) in  being more  likely to  terminate a
helical fragment. The G$\cdot$U  wobble pair (G$\cdot$U$_{w}$) is more
apt than either  of the canonical pairs to flank a  ``nick'' in an RNA
helix, and the U$\cdot$U wobble pair (U$\cdot$U$_{w}$) has an apparent
inability to stack at the  ends of a duplex.  The overall distribution
of  base pairs within  the RNA  helical arrays  resembles that  of the
dominant G$\cdot$C Watson-Crick pairs.  The ``nicks'' and inserts that
disrupt the organized structures  occur, on average, every fourth base
pair. The mean  length of the quasi-continuous helical  domains in the
dataset, obtained from the quotient of the number of base pairs in the
interior and the number at the  termini of the stacked arrays, is 11.3
bp.


3.4. Base-pair deformability

The  superposed   images  of  hydrogen-bonding  patterns   in  Fig.  3
illustrate  the  overall deformability  of  the base-pair  structures.
Table  4 quantifies  this information  in  terms of  the mean  values,
standard deviations, and volumes  of conformation space sampled by the
rigid-body  parameters  that  relate  the dominant  base  pairs.   The
variability is  also apparent in  the dispersion of  distances between
hydrogen-bond  donor  and acceptor  atoms  (Table  2).  The  composite
information  reveals an intrinsic  deformability in  the non-canonical
pairs beyond  that in the Watson-Crick  arrangements.  The enhancement
in deformation  stems from greater  variation in the  three rigid-body
parameters  -  Shear,  Stretch,  and  Opening  -  that  determine  the
hydrogen-bonding  patterns  (Table 4).  By  contrast, the  ``stiffer''
Watson-Crick pairs tend to  distort via Stagger, Propeller, and Buckle
- the three parameters that  control the planarity of associated bases
[15]. The hydrogen-bonding pattern is accordingly more variable in the
non-canonical  than   in  the   canonical  pairs.  Evidence   of  this
variability appears  in the fraction  of base pairs with  the expected
hydrogen-bond   pattern   (Table  2).   Whereas   $\sim$90\%  of   the
Watson-Crick  pairs  associate  through  the  expected  donor-acceptor
patterns,  the proportion  of  non-canonical pairs  with the  expected
patterns is lower, e.g., only 54\% of U$\cdot$U wobble pairs associate
through O2$\vdots$H-N3  and N3- H$\vdots$O4 interactions.  Some of the
non-canonical pairs are  ``melted'' in the sense that  they form fewer
hydrogen   bonds,   but    others   make   additional   donor-acceptor
contacts.  The  units  of  the  rigid-body  parameters,  i.e.,  angles
expressed  in  degrees and  distances  in  Ångstrom units,  complicate
assessment of the relative  deformabilities of different base pairs in
terms of the  accessible conformational volume V. A  larger value of V
does not  necessarily indicate  greater overall spatial  movement. The
root-mean-square  deviations  of  the  superimposed images  yield  the
following  deformability   rankings:  U$\cdot$U$_{w}$  (0.74   \AA)  >
G$\cdot$A$_{WC}$  (0.54 \AA)  $\approx$ G$\cdot$A$_{s}$  (0.50  \AA) >
G$\cdot$U$_{w}$  (0.45  \AA) $\approx$  A$\cdot$U$_{H}$  (0.43 \AA)  >
G$\cdot$C$_{WC}$ (0.38  \AA) $\approx$ A$\cdot$U$_{WC}$  (0.36 \AA), a
slightly  different ordering  from  the computed  values  of V  (Table
3). Surprisingly,  some of the  most deformable base pairs  occur with
greater likelihood  in the duplex interior e.g.,  the U$\cdot$U wobble
pair, and some of the more  rigid pairs occur at ``nicks'' or resemble
drug-like inserts,  e.g., the  A$\cdot$U Hoogsteen pair.  The relative
stiffness of  the latter arrangements  comes at no surprise  given the
preferential crystallization of free A  and U in Hoogsteen pairs [26].
Finally,  the  non-canonical  base  pairs  have  unique  deformability
profiles  partially  reflective of  the  pattern  of interaction.  For
example,  G  and  U tend  to  fluctuate  via  Shear along  the  short,
hydrogenbonded axis of the G$\cdot$U wobble  pair, but G and A tend to
slide  past one another  via Stretch  along the  long axis  of contact
between sheared  G$\cdot$A pairs.  The A$\cdot$U Hoogsteen  bases move
along both  axes and  rotate in the  base-pair plane via  Opening. The
major fluctuations of the  U$\cdot$U wobble and G$\cdot$A Watson-Crick
pairs entail Shear and Opening.







In the helical regions data we quantify:

Abundances (Counts)
Deformabilites
Helical Context




\section{Clustering of Yurong's Classification}

\bibliography{biblio}

