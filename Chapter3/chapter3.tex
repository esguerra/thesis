\chapter{RNA Base-Pairing}
\label{basepairs} 
\bibliographystyle{nar}
\section{Canonical and Noncanonical Base-pairs}
As  shown   in  Figure  \ref{fig:saenger28},  there   can  be  various
base-pairing patterns between heterocyclic  bases in nucleic acids due
to the  variety of possible  hydrogen bonding interactions.   The most
prevalent  hydrogen   bonding  pattern   is  that  of   the  canonical
Watson-Crick  base-pair.  All  other  possible patterns  are known  as
non-canonical base-pairs and  are more common in RNA  than in DNA.  We
used the 3DNA \cite{lu2003} software package to find all base-pairs in
a  non-redundant   dataset  of   RNA  structures  obtained   by  X-ray
crystallography with  resolution better than 3.5  \AA~ downloaded from
the  protein data  bank  (PDB).   We also  constrained  our search  to
helical  regions,  which  are  defined  as  having  three  consecutive
base-pairs  or  more  which  need  not be  covalently  bonded  by  the
sugar-phosphate     backbone     between    consecutive     base-pairs
\cite{olson2009}.

Our database is  non-redundant in the sense that  from the main source
of RNA structural information, which is the ribosome, we used only one
of the  available structures  per organism, that  is, one for  each of
\textit{Deinococus   radiodurans},   \textit{Haloarcula  marismortui},
\textit{Escherichi          coli},         and         \textit{Thermus
  thermophilus}. Table~\ref{tab:dbase}  shows in detail  the number of
bases per  RNA type in  our dataset. It's  interesting to see  that in
general the content of G and C, is higher than that of A and U.
\begin{table}[htbp]
\begin{center}
\begin{tabular}{|l|c|r|r|r|r|}
\hline
RNA Type & \multicolumn{1}{p{2cm}|}{Number of Structures} & \multicolumn{1}{c|}{G} &
\multicolumn{1}{c|}{C} & \multicolumn{1}{c|}{A} &
\multicolumn{1}{c|}{U} \\ \hline \hline
small helices & 78 & 891 & 753 & 404 & 442 \\ \hline
drug-RNA & 36 & 932 & 862 & 365 & 433 \\ \hline
protein-RNA & 207 & 4001 & 3457 & 1771 & 1731 \\ \hline
protein-tRNA & 9 & 175 & 155 & 98 & 87 \\ \hline
rRNA & 13 & 3866 & 2949 & 1939 & 1785 \\ \hline
tRNA & 13 & 205 & 159 & 124 & 112 \\ \hline
ribozyme & 113 & 2434 & 2086 & 1438 & 1150 \\ \hline
Total & 469 & \multicolumn{1}{c|}{12504} & \multicolumn{1}{c|}{10421} & \multicolumn{1}{c|}{6139} & \multicolumn{1}{c|}{5740} \\ \hline
\end{tabular}
\caption{Classification of RNA Types in Non-Redundant Dataset at less
  than 3.5 \AA~(For Base-Pairs in Helices of 3 base-pairs or more).}
\label{tab:dbase}
\end{center}
\end{table}

We classified the RNA base-pairs  in our dataset using three criteria.
(1) The Leontis-Westhof edge classification scheme \cite{leontis1998}.
(2)  The  six  rotational  and translational  base-pairing  rigid-body
parameters,  Shear, Stretch, Stagger,  Buckle, Propeller  and Opening.
(3) The location of base-pairs  in helices, that is, their location in
``intact''   covalently  bonded   backbones,  and   in  the   ends  of
``quasi-continuous'' backbones.



In the helical regions data we quantify:

Abundances (Counts)
Deformabilites
Helical Context




\section{Clustering of Yurong's Classification}

\bibliography{biblio}

