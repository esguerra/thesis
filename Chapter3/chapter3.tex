\chapter{RNA Base-Pairs}
\label{basepairs} 
\bibliographystyle{nar}
\section{Canonical and Noncanonical Base-pairs}
As  shown   in  Figure  \ref{fig:saenger28},  there   can  be  various
base-pairing patterns between heterocyclic  bases in nucleic acids due
to the  variety of possible  hydrogen bonding interactions.   The most
prevalent hydrogen  bonding pattern  in nucleic acids  is that  of the
canonical  Watson-Crick  (WC)  base-pair.   Other  possible  patterns
besides the WC one are known as non-canonical base-pairs and are more
common in  RNA than in DNA.   We used the  3DNA \cite{lu2003} software
package  to find  all base-pairs  in  a non-redundant  dataset of  RNA
structures  obtained by X-ray  crystallography with  resolution better
than 3.5  \AA~ downloaded from the  protein data bank  (PDB).  We also
constrained our search to helical regions, which are defined as having
three  consecutive base-pairs  or more  which need  not  be covalently
bonded by the  sugar-phosphate backbone between consecutive base-pairs
\cite{olson2009}.

Our dataset is  non-redundant in the sense that  from the main source
of RNA structural information, which is the ribosome, we used only one
of the  available structures  per organism, that  is, one for  each of
\textit{Deinococus   radiodurans},   \textit{Haloarcula  marismortui},
\textit{Escherichi          coli},         and         \textit{Thermus
  thermophilus}. Table~\ref{tab:dbase}  shows in detail  the number of
bases per  RNA type in  our dataset. It's  interesting to see  that in
general the content of  G and C, is higher than that  of A and U which
might be due to the  highest overall stability of G$\cdot$C base-pairs
against A$\cdot$U base-pairs.

\begin{table}[htbp]
\begin{center}
\begin{tabular}{|l|c|r|r|r|r|}
\hline
RNA Type & \multicolumn{1}{p{2cm}|}{Number of Structures} & \multicolumn{1}{c|}{G} &
\multicolumn{1}{c|}{C} & \multicolumn{1}{c|}{A} &
\multicolumn{1}{c|}{U} \\ \hline \hline
small helices & 78 & 891 & 753 & 404 & 442 \\ \hline
drug-RNA & 36 & 932 & 862 & 365 & 433 \\ \hline
protein-RNA & 207 & 4001 & 3457 & 1771 & 1731 \\ \hline
protein-tRNA & 9 & 175 & 155 & 98 & 87 \\ \hline
rRNA & 13 & 3866 & 2949 & 1939 & 1785 \\ \hline
tRNA & 13 & 205 & 159 & 124 & 112 \\ \hline
ribozyme & 113 & 2434 & 2086 & 1438 & 1150 \\ \hline
Total & 469 & \multicolumn{1}{c|}{12504} & \multicolumn{1}{c|}{10421} & \multicolumn{1}{c|}{6139} & \multicolumn{1}{c|}{5740} \\ \hline
\end{tabular}
\caption{Classification of RNA types in non-redundant dataset at less
  than 3.5 \AA~(for base-pairs in helices of 3 base-pairs or more).}
\label{tab:dbase}
\end{center}
\end{table}

In Table  \ref{tab:bpcomp} we show  all possible base-pairs  formed by
unmodified nucleotides in our dataset,  it's clear from this table that
the G$\cdot$C  and A$\cdot$U base-pairs  dominate widely the  space of
possible  RNA base-pairs  in helical  regions  making up  80\% of  all
base-pairs  and  if  we  only  count those  that  form  canonical  WC
base-pairs (9500  G$\cdot$C, and 3069  A$\cdot$U), they make  up about
73\% of all base-pairs in helical regions.

\begin{table}[htbp]
\begin{center}
\begin{tabular}{|c|c|c|c|c|}
\hline
A    &      G    &      C    &      U    &      B$\cdot$B' \\ \hline \hline
384  &    980    &    313    &   3975    &      A  \\ \hline
     &    128    &   9913    &   1282    &      G  \\ \hline
     &           &     63    &    103    &      C  \\ \hline
     &           &           &    187    &      U  \\ \hline
\end{tabular}
\caption{Composition  of  base-pairs  in  non-redundant  dataset  with
  resolutions better  than 3.5  \AA~ (for base-pairs  in helices  of 3
  base-pairs or more).  Note that 9500 out of  9913 G$\cdot$C and 3069
  out of 3975 A$\cdot$U are canonical WC base-pairs}
\label{tab:bpcomp}
\end{center}
\end{table}

As will  be show later (Table~\ref{tab:seven}),  a considerable amount
of A$\cdot$U's  pair in the  Hoogsteen fashion. Few of  these examples
form    U$\cdot$A$\cdot$U   triplets   containing    a   WC    and   a
Hoogsteen \footnote{Hoogsteen base-pairs  are illustrated by structure
  XXIII  of  the Saenger  classification  of  base-pairs  as shown  in
  Figure~\ref{fig:saenger28}} base-pair in RNA helical regions.

\subsection{RNA Base-Pairs Classification}
We classified the RNA base-pairs  in our dataset using three criteria.
(1) The Leontis-Westhof  edge classification scheme \cite{leontis1998}
which is based on the  identification of three major interacting edges
for hydrogen-bond  formation called WC  edge (W), Hoogsteen  edge (H),
and Sugar edge (S).  (2) The rotational and translational base-pairing
rigid-body parameters, Shear,  Stretch, Stagger, Buckle, Propeller and
Opening.  (3)  The location of  base-pairs in helices, that  is, their
location  in ``intact''  covalently bonded  sugar-phosphate backbones,
and in  the ends  of ``quasi-continuous'' helices  with breaks  in the
sugar-phosphate backbone.

In  out dataset  we  find that  $\sim$90\%  of the  base-pairs in  RNA
helices will form base-pairs in one of seven possible hydrogen-bonding
types, these  were found to  be canonical WC G$\cdot$C  and A$\cdot$U,
and the  non-canonical G$\cdot$U wobble,  sheared G$\cdot$A, Hoogsteen
A$\cdot$U, WC type G$\cdot$A, and U$\cdot$U wooble base-pair. Detailed
results  showing how  these  seven major  RNA  base-pairing types  are
classified  according to  various schemes,  and the  details  of their
hydrogen-bond  distances  is  given  in Table~\ref{tab:seven}  and  in
Figure~\ref{fig:pairs}

\begin{sidewaystable}
\begin{center}
\begin{tabular}{|c c|c c|c|c|c c|c|}
\hline
Base-pair & & Hydrogen bonds &  & Sign & Saenger & Leontis-Westhof & &
Number \\
\hline
\hline
\multicolumn{9}{|l|}{Canonical} \\
\hline
G$\cdot$C & Watson-Crick & N2-H$\cdots$O2 & 2.79(0.17) & - & XIX & cis
 & W/W & 9500$_{\text{x0.90}}$ \\
 & & O6$\cdots$H-N4 & 2.92(0.18) & & & & &  \\
 & & N1-H$\cdots$N3 & 2.89(0.13) & & & & &  \\
\hline
A$\cdot$U & Watson-Crick & N1$\cdots$H-N3 & 2.84(0.14) & - & XX & cis
& W/W & 3069$_{\text{x0.93}}$ \\
 & & N6-H$\cdots$O4 & 2.97(0.18) & & & & &  \\
\hline
\multicolumn{9}{|l|}{Non-canonical} \\
\hline
G$\cdot$U & Wobble & N1-H$\cdots$O2 & 2.79(0.16) & - & XXVIII & cis
 & W/W & 1049$_{\text{x0.69}}$ \\
 & & O6$\cdots$H-N3 & 2.85(0.16) & & & & &  \\
\hline
G$\cdot$A & Sheared & N2-H$\cdots$N7 & 2.89(0.17) & + & XI & trans
 & H/S & 509$_{\text{x0.59}}$ \\
 & & N3$\cdots$H-N6 & 3.03(0.18) & & & & &  \\
\hline
A$\cdot$U & Hoogsteen & N6-H$\cdots$O2 & 2.91(0.21) & + & XXIII & trans
 & H/W & 354$_{\text{x0.71}}$ \\
 & & N7$\cdots$H-N3 & 2.90(0.17) & & & & &  \\
\hline
G$\cdot$A & Watson-Crick & N1-H$\cdots$N1 & 2.84(0.17) & - & VIII & cis
 & W/W & 185$_{\text{x0.85}}$ \\
 & & O6$\cdots$H-N6 & 2.91(0.20) & & & & &  \\
\hline
U$\cdot$U & Wobble & O2$\cdots$H-N3 & 2.95(0.24) & - & XVI & cis
 & W/W & 141$_{\text{x0.54}}$ \\
 & & N3$\cdots$H-O4 & 2.87(0.15) & & & & &  \\
\hline
\end{tabular}
\caption{Seven main base-pairing types in RNA helical dataset.}
\label{tab:seven}
\end{center}  
\end{sidewaystable}  

\begin{figure}[htbp]
\centering
\includegraphics[scale=1.1]{Chapter3/sevenpairs.png}
\caption{Seven   most   prominent  pairs   in   RNA  helical   regions
  dataset.  The  images  to  the  left  of  each  base-pair  show  the
  hydrogen-bond  connectivity  as magenta  colored  dashed lines.  The
  rigth  side   images  on   each  base-pair  representation   show  a
  superposition of the base-pairs  in our helical dataset, centered in
  the middle base triad reference frame. }
\label{fig:pairs}
\end{figure}

\subsection{Base-Pairs in Helical Regions}
Our classification also includes the location of base-pairs in helical
regions, that is, whether they are  in the interior or in the terminal
ends     of     ``intact''     or     ``quasi-continuous''     helical
regions.  Figure~\ref{fig:helregxin}  illustrates  the  two  types  of
helical regions  mentioned. In (a) an  ``intact'' helical region
is depicted and in (b) a ``quasi-continuos'' helical region is shown.

\begin{figure}
\centering
\includegraphics[scale=0.4]{Chapter3/helcontext.png}
\caption{(a)  Intact  and  (b)  quasi-continuous  helical  regions  in
  RNA. Image kindly provided by Dr. Yurong Xin.}
\label{fig:helregxin}
\end{figure}  

The results obtained from  the classification of base-pairs in helical
regions  are  shown in  Table~\ref{tab:helcontext}.   We  see that  in
particular  the   A$\cdot$U  Hoogsteen  pair  (A$\cdot$U$_{\text{H}}$)
stands out as  being present mainly as an  insert \footnote{Similar to
  the way  intercalating ligands (intercalators)  insert themselves in
  DNA.} in helical regions, sometimes  in nicked regions and rarely in
intact   ones.   Now,   for  the   G$\cdot$A   Watson-Crick-like  pair
(G$\cdot$A$_{WC}$)  we  see  the  inverse  situation to  that  of  the
A$\cdot$U  Hoogsteen pair,  that is,  it rarely  occurs as  an insert,
prefering to be  in nicks, and sometimes in  intact regions, a similar
situation     happens    with     the    sheared     G$\cdot$A    pair
(G$\cdot$A$_{\text{s}}$),  which is also  rare as  an insert  and it's
mainly found in intact regions, and sometimes in nicks.  The canonical
WC G$\cdot$C base-pair is two times more likely to occur at the end of
helical  regions than  the other  canonical base-pair  A$\cdot$U.  The
helical  context of  the  G$\cdot$U wobble  pair (G$\cdot$U$_{w}$)  is
quite  similar  to that  of  the  canonical  base-pairs, more  closely
resembling  the  context  of  G$\cdot$C$_{\text{WC}}$,  than  that  of
A$\cdot$U$_{\text{WC}}$, it differentiates from them on being slightly
more prevalent  in nicked regions  than any of  the two, and  the G$\cdot$U
wobble pair (G$\cdot$U$_{\text{w}}$) is present in a similar context to
that of the canonical A$\cdot$U, more commonly seen in interior regions
than    in   ends.    It   is    interesting   to    see    that   the
G$\cdot$C$_{\text{WC}}$ has a helical context practically identical to
the one given for all base-pairs.

The mean  length of the helical  domains in the dataset is 11 base-pairs. 

\begin{table}[htbp]
\begin{center}
\begin{tabular}{|c|c|c|c|c|c|c|c|c|}
\hline
Helical context & \multicolumn{8}{c|}{Base-pair} \\
\hline
 & All & G$\cdot$C$_{\text{WC}}$ & A$\cdot$U$_{\text{WC}}$ &
G$\cdot$U$_{\text{w}}$ & G$\cdot$A$_{\text{s}}$ &
A$\cdot$U$_{\text{H}}$ & G$\cdot$A$_{\text{WC}}$ &
U$\cdot$U$_{\text{w}}$  \\
\hline
\multicolumn{9}{|c|}{Interior} \\
\hline
Intact &  0.62 & 0.62 & 0.75 & 0.63 & 0.34 & 0.05 & 0.25 & 0.74 \\
Nick   &  0.20 & 0.20 & 0.16 & 0.26 & 0.25 & 0.29 & 0.66 & 0.13 \\
Insert &  0.02 & 0.01 & 0.01 & 0.00 & 0.00 & 0.42 & 0.01 & 0.03 \\
\hline
\multicolumn{9}{|c|}{Ends} \\
\hline
Intact &  0.13 & 0.15 & 0.07 & 0.10 & 0.33 & 0.05 & 0.06 & 0.10 \\
Insert &  0.02 & 0.01 & 0.01 & 0.01 & 0.05 & 0.19 & 0.02 & 0.00 \\
\hline
\end{tabular}
\caption{Distribution in helical context of the seven most abundant
  base-pairs in our RNA helical regions dataset.}
\label{tab:helcontext}
\end{center}
\end{table}





3.4. Base-pair deformability

The  superposed   images  of  hydrogen-bonding  patterns   in  Fig.  3
illustrate  the  overall deformability  of  the base-pair  structures.
Table  4 quantifies  this information  in  terms of  the mean  values,
standard deviations, and volumes  of conformation space sampled by the
rigid-body  parameters  that  relate  the dominant  base  pairs.   The
variability is  also apparent in  the dispersion of  distances between
hydrogen-bond  donor  and acceptor  atoms  (Table  2).  The  composite
information  reveals an intrinsic  deformability in  the non-canonical
pairs beyond  that in the Watson-Crick  arrangements.  The enhancement
in deformation  stems from greater  variation in the  three rigid-body
parameters  -  Shear,  Stretch,  and  Opening  -  that  determine  the
hydrogen-bonding  patterns  (Table 4).  By  contrast, the  ``stiffer''
Watson-Crick pairs tend to  distort via Stagger, Propeller, and Buckle
- the three parameters that  control the planarity of associated bases
[15]. The hydrogen-bonding pattern is accordingly more variable in the
non-canonical  than   in  the   canonical  pairs.  Evidence   of  this
variability appears  in the fraction  of base pairs with  the expected
hydrogen-bond   pattern   (Table  2).   Whereas   $\sim$90\%  of   the
Watson-Crick  pairs  associate  through  the  expected  donor-acceptor
patterns,  the proportion  of  non-canonical pairs  with the  expected
patterns is lower, e.g., only 54\% of U$\cdot$U wobble pairs associate
through O2$\cdots$H-N3  and N3- H$\cdots$O4 interactions.  Some of the
non-canonical pairs are  ``melted'' in the sense that  they form fewer
hydrogen   bonds,   but    others   make   additional   donor-acceptor
contacts.  The  units  of  the  rigid-body  parameters,  i.e.,  angles
expressed  in  degrees and  distances  in  Ångstrom units,  complicate
assessment of the relative  deformabilities of different base pairs in
terms of the  accessible conformational volume V. A  larger value of V
does not  necessarily indicate  greater overall spatial  movement. The
root-mean-square  deviations  of  the  superimposed images  yield  the
following  deformability   rankings:  U$\cdot$U$_{w}$  (0.74   \AA)  >
G$\cdot$A$_{WC}$  (0.54 \AA)  $\approx$ G$\cdot$A$_{s}$  (0.50  \AA) >
G$\cdot$U$_{w}$  (0.45  \AA) $\approx$  A$\cdot$U$_{H}$  (0.43 \AA)  >
G$\cdot$C$_{WC}$ (0.38  \AA) $\approx$ A$\cdot$U$_{WC}$  (0.36 \AA), a
slightly  different ordering  from  the computed  values  of V  (Table
3). Surprisingly,  some of the  most deformable base pairs  occur with
greater likelihood  in the duplex interior e.g.,  the U$\cdot$U wobble
pair, and some of the more  rigid pairs occur at ``nicks'' or resemble
drug-like inserts,  e.g., the  A$\cdot$U Hoogsteen pair.  The relative
stiffness of  the latter arrangements  comes at no surprise  given the
preferential crystallization of free A  and U in Hoogsteen pairs [26].
Finally,  the  non-canonical  base  pairs  have  unique  deformability
profiles  partially  reflective of  the  pattern  of interaction.  For
example,  G  and  U tend  to  fluctuate  via  Shear along  the  short,
hydrogenbonded axis of the G$\cdot$U wobble  pair, but G and A tend to
slide  past one another  via Stretch  along the  long axis  of contact
between sheared  G$\cdot$A pairs.  The A$\cdot$U Hoogsteen  bases move
along both  axes and  rotate in the  base-pair plane via  Opening. The
major fluctuations of the  U$\cdot$U wobble and G$\cdot$A Watson-Crick
pairs entail Shear and Opening.







In the helical regions data we quantify:

Abundances (Counts)
Deformabilites
Helical Context




\section{Clustering of Yurong's Classification}

\bibliography{biblio}

