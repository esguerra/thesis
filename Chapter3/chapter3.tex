\chapter{RNA Base-Pairing}
\label{basepairs} 
\bibliographystyle{nar}
\section{Canonical and Noncanonical Base-pairs}
As  seen   in  Figure   \ref{fig:saenger28},  there  can   be  various
base-pairing patterns between heterocyclic  bases in nucleic acids due
to  a variety  of possible  hydrogen bonding  interactions.   The most
prevalent   hydrogen   bonding   pattern   is   known   as   canonical
Watson-Crick, all  other possible patterns are  known as non-canonical
base-pairs and  are more common in RNA  than in DNA.  We  used 3DNA to
find all  base-pairs in a  non-redundant database of  X-ray determined
RNA structures from the PDB with resolutions less than or equal to 3.5
\AA.  We also  constrained our search to helical  regions in RNA. Such
helical regions are composed of  3 consecutive base-pairs or more, and
they  need not be  covalently bonded  by the  sugar-phosphate backbone
between  consecutive  base-pairs.   For  more details  the  reader  is
refered to Olson et al. \cite{olson2009}.



In the helical regions data we quantify:

Abundances (Counts)
Deformabilites
Helical Context



NON-REDUNDANT DATABASE AND CONSTRAIN TO HELICAL REGIONS.

We use a non-redundant dataset of RNA structures.  By non-redundant we
mean to say  that, for the main source  of RNA structural information,
which is  the ribosome, we used  only one of  the available structures
per   organism,  that   is,   one  for   each  of   \textit{Deinococus
  Radiodurans},  \textit{Haloarcula  marismortui},  \textit{Escherichi
  coli},  and  \textit{Thermus thermophilus}.

\begin{table}[htbp]
\begin{center}
\begin{tabular}{|l|c|r|r|r|r|}
\hline
RNA Type & \multicolumn{1}{l|}{Counts} & \multicolumn{1}{c|}{G} &
\multicolumn{1}{c|}{C} & \multicolumn{1}{c|}{A} &
\multicolumn{1}{c|}{U} \\ \hline \hline
small helices & 78 & 891 & 753 & 404 & 442 \\ \hline
drug-RNA & 36 & 932 & 862 & 365 & 433 \\ \hline
protein-RNA & 207 & 4001 & 3457 & 1771 & 1731 \\ \hline
protein-tRNA & 9 & 175 & 155 & 98 & 87 \\ \hline
rRNA & 13 & 3866 & 2949 & 1939 & 1785 \\ \hline
tRNA & 13 & 205 & 159 & 124 & 112 \\ \hline
ribozyme & 113 & 2434 & 2086 & 1438 & 1150 \\ \hline
Total & 469 & \multicolumn{1}{c|}{12504} & \multicolumn{1}{c|}{10421} & \multicolumn{1}{c|}{6139} & \multicolumn{1}{c|}{5740} \\ \hline
\end{tabular}
\caption{Classification of RNA Types in Non-Redundant Dataset at less
  than 3.5 \AA~(For Base-Pairs in Helices of 3 base-pairs or more).}
\label{dbase}
\end{center}
\end{table}




\section{Clustering of Yurong's Classification}

\bibliography{biblio}

