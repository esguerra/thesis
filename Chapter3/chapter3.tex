\chapter{RNA Base-Pairing}
\label{basepairs} 
\bibliographystyle{nar}
\section{Canonical and Noncanonical Base-pairs}
In general as seen in chapter I figure, there can be various
base-pairing patterns due to a variety of preferred hydrogen bonding
configurations. Such configurations are rarely seen in DNA, and are
more prevalent in RNA.
We see how much data is available. The reason to make a contability of
available data is that we construct polymer models based on mean
values of properties. If there is enough data, then we can model,
without it the models can't be expanded. So this is the first step.

We have collected data in RNA helical regions in non-redundant
database.
What we mean by non-redundant database.
What we mean by helical regions.

In the helical regions data we quantify:

Abundances (Counts)
Deformabilites
Helical Context
P-atoms positions


NON-REDUNDANT DATABASE AND CONSTRAIN TO HELICAL REGIONS.

We use a non-redundant dataset of RNA structures.  By non-redundant we
mean to say  that, for the main source  of RNA structural information,
which is  the ribosome, we used  only one of  the available structures
per   organism,  that   is,   one  for   each  of   \textit{Deinococus
  Radiodurans},  \textit{Haloarcula  marismortui},  \textit{Escherichi
  coli},  and  \textit{Thermus thermophilus}.



\begin{table}[htbp]
\begin{center}
\begin{tabular}{|l|c|r|r|r|r|}
\hline
RNA Type & \multicolumn{1}{l|}{Counts} & \multicolumn{1}{c|}{G} &
\multicolumn{1}{c|}{C} & \multicolumn{1}{c|}{A} &
\multicolumn{1}{c|}{U} \\ \hline \hline
small helices & 78 & 891 & 753 & 404 & 442 \\ \hline
drug-RNA & 36 & 932 & 862 & 365 & 433 \\ \hline
protein-RNA & 207 & 4001 & 3457 & 1771 & 1731 \\ \hline
protein-tRNA & 9 & 175 & 155 & 98 & 87 \\ \hline
rRNA & 13 & 3866 & 2949 & 1939 & 1785 \\ \hline
tRNA & 13 & 205 & 159 & 124 & 112 \\ \hline
ribozyme & 113 & 2434 & 2086 & 1438 & 1150 \\ \hline
Total & 469 & \multicolumn{1}{c|}{12504} & \multicolumn{1}{c|}{10421} & \multicolumn{1}{c|}{6139} & \multicolumn{1}{c|}{5740} \\ \hline
\end{tabular}
\caption{Classification of RNA Types in Non-Redundant Dataset at less
  than 3.5 \AA~(For Base-Pairs in Helices of 3 base-pairs or more).}
\label{dbase}
\end{center}
\end{table}




\section{Clustering of Yurong's Classification}

\bibliography{biblio}

