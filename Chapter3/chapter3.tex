\chapter{RNA Base-Pairs}
\label{basepairs} 
\bibliographystyle{nar}
\section{Canonical and Non-canonical Base-pairs}
As  shown  in  Figure  \ref{fig:saenger28},  there  is  a  variety  of
base-pairing patterns between the  heterocyclic bases in nucleic acids
due  to the  many possible  hydrogen bonding  interactions.   The most
prevalent hydrogen  bonding pattern  in nucleic acids  is that  of the
canonical  Watson-Crick (WC)  base-pair.  Patterns  other than  the WC
pairs are known as non-canonical base pairs and are more common in RNA
than in DNA.  We used  the 3DNA \cite{lu2003} software package to find
all base  pairs in a  non-redundant dataset of RNA  crystal structures
with resolution better than 3.5  \AA~ downloaded from the Protein Data
Bank (PDB).  We also constrained  our search to helical regions, which
are  defined  as stretches  of  three  or  more spatially  consecutive
base-pairs which need not  be covalently bonded by the sugar-phosphate
backbone between sequentially consecutive base pairs \cite{olson2009}.

\begin{table}[htbp]
\begin{center}
\begin{tabular}{|l|c|r|r|r|r|}
\hline
RNA Type & \multicolumn{1}{p{2cm}|}{Number of Structures} & \multicolumn{1}{c|}{G} &
\multicolumn{1}{c|}{C} & \multicolumn{1}{c|}{A} &
\multicolumn{1}{c|}{U} \\ \hline 
small helices & 78 & 36 & 30 & 16 & 18 \\ \hline
drug-RNA & 36 & 36 & 33 & 14 & 17 \\ \hline
protein-RNA & 207 & 37 & 32 & 16 & 16 \\ \hline
protein-tRNA & 9 & 34 & 30 & 19 & 17 \\ \hline
rRNA & 13 & 37 & 28 & 18 & 17 \\ \hline
tRNA & 13 & 34 & 27 & 21 & 19 \\ \hline
ribozyme & 113 & 34 & 29 & 20 & 16 \\ \hline
Total & 469 & \multicolumn{1}{c|}{36} & \multicolumn{1}{c|}{30} & \multicolumn{1}{c|}{18} & \multicolumn{1}{c|}{16} \\ \hline
\end{tabular}
\caption{Description  of the  types  of RNA  structures  and the  base
  content of  each group in  the non-redundant dataset of  RNA crystal
  structures with resolution better than 3.5 \AA given as a percentage
  of  the total  number of  structures  per group.   The listed  bases
  comprise  the base  pairs in  helices of  three or  more base-pairs.
  Further detail on the structures which compose the dataset including
  PDB\_ID's  and NDB\_ID's  can  be obtained  online as  supplementary
  material attached to Olson et al. \cite{olson2009} results.}
\label{tab:dbase}
\end{center}
\end{table}

Our dataset is non-redundant in the sense that from the main source of
RNA structural information, which is the ribosome, we used only one of
the  available structures  per  organism,  that is,  one  for each  of
\textit{Deinococus   radiodurans},   \textit{Haloarcula  marismortui},
\textit{Escherichi          coli},         and         \textit{Thermus
  thermophilus}. Table~\ref{tab:dbase}  shows in detail  the number of
bases for each  RNA type in our dataset of  helical structures.  It is
interesting to see  that in general the content of G  and C, is higher
than that  of A  and U. The  difference might  be related to  a higher
overall  stability  of  G$\cdot$C  base-pairs  compared  to  A$\cdot$U
base-pairs.

In  Table  \ref{tab:bpcomp}  we  show  the  number  of  base-pairs  of
different  chemical  types formed  by  unmodified  nucleotides in  our
dataset,  it is  clear from  the  table that  G$\cdot$C and  A$\cdot$U
base-pairs  dominate the  RNA  base-pairs formed  in helical  regions,
making up 80\% of all the base-pairs. If we count only those that form
canonical  WC base-pairs  (9500  G$\cdot$C, and  3069 A$\cdot$U),  the
number corresponds to 73\% of all base-pairs in helical regions.

As will  be show later (Table~\ref{tab:seven}),  a considerable number
of the A$\cdot$U's  pairs associate in a Hoogsteen  arrangement. A few
of these examples form  U$\cdot$A$\cdot$U triplets containing a WC and
a   Hoogsteen  \footnote{Hoogsteen   base-pairs  are   illustrated  by
  structure XXIII of the Saenger classification of base-pairs as shown
  in Figure~\ref{fig:saenger28}} base pair in the RNA helical regions.

\begin{table}[htbp]
\begin{center}
\begin{tabular}{|c|c|c|c|c|}
\hline
A    &      G    &      C    &      U    &      B/B$'$ \\ \hline
384  &    980    &    313    &   3975    &      A  \\ \hline
     &    128    &   9913    &   1282    &      G  \\ \hline
     &           &     63    &    103    &      C  \\ \hline
     &           &           &    187    &      U  \\ \hline
\end{tabular}
\caption{Composition  of base  pairs in  the  non-redundant structural
  dataset. Note that 9500 out of 9913 G$\cdot$C and 3069
  out of  3975 A$\cdot$U  are canonical WC  base pairs. See  legend to
  Table \ref{tab:dbase} for dataset details.}
\label{tab:bpcomp}
\end{center}
\end{table}

\subsection{RNA Base-Pairs Classification}
We classified the RNA base  pairs in our dataset using three criteria:
(1) the Leontis-Westhof edge classification scheme \cite{leontis1998},
which is based on the  identities of the three major interacting edges
for  hydrogen-bond formation  called the  WC (W),  Hoogsteen  (H), and
Sugar  (S)  edges, (2)  the  rotational  and translational  rigid-body
base-pairing  parameters  called,  shear,  stretch,  stagger,  buckle,
propeller and opening,  and (3) the location of  the base-pairs within
the helices,  that is, their location in  either ``intact'' covalently
bonded   sugar-phosphate  backbones  or   within  ``quasi-continuous''
helices  with  breaks  in   the  sugar-phosphate  backbone  and  their
positions within these kinds of helices (see below).

We find  that $\sim$90\% of the base  pairs in the RNA  helices in our
dataset  form base  pairs in  one of  seven  possible hydrogen-bonding
types; canonical  WC G$\cdot$C and A$\cdot$U  pairs, and non-canonical
G$\cdot$U  wobble,  sheared G$\cdot$A,  Hoogsteen  A$\cdot$U, WC  type
G$\cdot$A, and  U$\cdot$U wooble base pairs.  Detailed results showing
how these seven major  RNA base-pairing types are classified according
to various  schemes, and the details of  their hydrogen-bond distances
is given in Table~\ref{tab:seven} and in Figure~\ref{fig:pairs}

\begin{figure}[ht]
\centering
\includegraphics[angle=-90, scale=0.75]{Chapter3/sevenpairs.png}
\caption{Seven most prominent base pairs in RNA helical regions in our
  structural  dataset  shown  in  images (a-g),  (a-b)  the  canonical
  G$\cdot$C and A$\cdot$U Watson-Crick pairs, (c) the wobble G$\cdot$U
  pair, (d) the wobble U$\cdot$U pair, (e) the sheared G$\cdot$A pair,
  (f)  the Watson-Crick-like  G$\cdot$A  pair, and  (g) the  Hoogsteen
  A$\cdot$U pair.  The images on  the left for each base-pair show the
  identities  of the  bases, atom  types (oxygen  red,  nitrogen blue,
  carbon  and   hydrogen  white,  and  C1$'$  atoms   gray),  and  the
  hydrogen-bond  connectivity  (magenta  colored dashed  lines).   The
  rigth  side   images  of   each  base-pair  representation   show  a
  superposition of the base pairs  in our helical dataset, centered in
  the middle  base triad  (MBT) reference frame  (The definition  of a
  middle base triad  is completely analogous to that  of a middle step
  triad as explained thoroughly in Appendix \ref{appendix_1a}).}
\label{fig:pairs}
\end{figure}

\begin{sidewaystable}
\begin{center}
\begin{tabular}{|c c|c c|c|c|c c|c|}
\hline
Base-pair & & Hydrogen bonds &  & Sign & Saenger & Leontis-Westhof & &
Number \\
\hline
\hline
\multicolumn{9}{|l|}{Canonical} \\
\hline
G$\cdot$C & Watson-Crick & N2-H$\cdots$O2 & 2.79(0.17) & - & XIX & cis
 & W/W & 9500$_{\text{x0.90}}$ \\
 & & O6$\cdots$H-N4 & 2.92(0.18) & & & & &  \\
 & & N1-H$\cdots$N3 & 2.89(0.13) & & & & &  \\
\hline
A$\cdot$U & Watson-Crick & N1$\cdots$H-N3 & 2.84(0.14) & - & XX & cis
& W/W & 3069$_{\text{x0.93}}$ \\
 & & N6-H$\cdots$O4 & 2.97(0.18) & & & & &  \\
\hline
\multicolumn{9}{|l|}{Non-canonical} \\
\hline
G$\cdot$U & Wobble & N1-H$\cdots$O2 & 2.79(0.16) & - & XXVIII & cis
 & W/W & 1049$_{\text{x0.69}}$ \\
 & & O6$\cdots$H-N3 & 2.85(0.16) & & & & &  \\
\hline
G$\cdot$A & Sheared & N2-H$\cdots$N7 & 2.89(0.17) & + & XI & trans
 & H/S & 509$_{\text{x0.59}}$ \\
 & & N3$\cdots$H-N6 & 3.03(0.18) & & & & &  \\
\hline
A$\cdot$U & Hoogsteen & N6-H$\cdots$O2 & 2.91(0.21) & + & XXIII & trans
 & H/W & 354$_{\text{x0.71}}$ \\
 & & N7$\cdots$H-N3 & 2.90(0.17) & & & & &  \\
\hline
G$\cdot$A & Watson-Crick & N1-H$\cdots$N1 & 2.84(0.17) & - & VIII & cis
 & W/W & 185$_{\text{x0.85}}$ \\
 & & O6$\cdots$H-N6 & 2.91(0.20) & & & & &  \\
\hline
U$\cdot$U & Wobble & O2$\cdots$H-N3 & 2.95(0.24) & - & XVI & cis
 & W/W & 141$_{\text{x0.54}}$ \\
 & & N3$\cdots$H-O4 & 2.87(0.15) & & & & &  \\
\hline
\end{tabular}
\caption{Seven  dominant  base-pairing  types  found  in  RNA  helical
  regions.   The  first  column  lists the  Gutell  and  collaborators
  nomenclature \cite{lee2004} of the  base pairs. Column two shows the
  standard hydrogen  bonding pattern  associated with each  base pair.
  Column  three displays  a negative  sign  if the  bases forming  the
  base-pair oppose  each other and a  positive sign if  they share the
  same   face   \cite{lu2003}.    Column   four  gives   the   Saenger
  classification as in  Figure \ref{fig:saenger28}.  Column five lists
  the Leontis-Westhof edge  classification obtained trough the RNAVIEW
  program \ref{yang2003},  and the last column gives  the total number
  of  identified base  pairs of  each category  and the  percentage of
  those which  comply exactly with the  hydrogen-bonding pattern shown
  in column two.}
\label{tab:seven}
\end{center}  
\end{sidewaystable}  


\subsection{Base-Pairs in Helical Regions}
Our classification also includes the locations of the base pairs in helical
regions, that is,  whether they are in the interior or  at the ends of
``intact'' or ``quasi-continuous'' helical
regions.  Figure~\ref{fig:helregxin}  illustrates  the  two  types  of
helical regions  mentioned. The ``intact'' helical region
is depicted in Figure \ref{fig:helregxin}a.
The ``quasi-continuous'' helical region shown in Figure \ref{fig:helregxin}b.

\begin{figure}
\centering
\includegraphics[scale=0.4]{Chapter3/helcontext.png}
\caption{(a)  Intact  and  (b)  quasi-continuous  helical  regions  in
  RNA. Image kindly provided by Dr. Yurong Xin.}
\label{fig:helregxin}
\end{figure}  

The locations of  the base pairs in helical  regions are summarized in
Table~\ref{tab:helcontext}.     The     A$\cdot$U    Hoogsteen    pair
(A$\cdot$U$_{\text{H}}$)  stands out  as  being present  mainly as  an
insert   \footnote{Similar   to    the   way   intercalating   ligands
  (intercalators)  insert  themselves  in  DNA.} in  helical  regions,
sometimes in nicked  regions and rarely in intact  ones. The G$\cdot$A
Watson-Crick-like  pair  (G$\cdot$A$_{\text{WC}}$)  differs  from  the
A$\cdot$U  Hoogsteen pair,  in that,  it rarely  occurs as  an insert,
preferring to be in nicks,  and sometimes in intact regions. A similar
situation     happens    with     the    sheared     G$\cdot$A    pair
(G$\cdot$A$_{\text{s}}$), which also rarely occurs as an insert and is
mainly found in intact regions, and sometimes in nicks.  The canonical
WC G$\cdot$C base-pair is two times more likely to occur at the end of
helical  regions than  the other  canonical base-pair  A$\cdot$U.  The
helical context of  the G$\cdot$U wobble pair (G$\cdot$U$_{\text{w}}$)
is quite  similar to  that of the  canonical base-pairs,  more closely
resembling  the  context  of  G$\cdot$C$_{\text{WC}}$,  than  that  of
A$\cdot$U$_{\text{WC}}$, it differentiates from them on being slightly
more  prevalent  in  nicked regions  than  any  of  the two,  and  the
G$\cdot$U wobble pair (G$\cdot$U$_{\text{w}}$) is present in a similar
context  to that  of the  canonical A$\cdot$U,  more commonly  seen in
interior  regions than in  ends.  It  is interesting  to see  that the
G$\cdot$C$_{\text{WC}}$ has a helical context practically identical to
the  one given for  all base-pairs.   The mean  length of  the helical
domains   in  the   dataset   is   11  bp   as   show  in   supplement
Figure~\ref{fig:hellength}  which indicates  that the  canonical A-RNA
conformation is predominant and is most likely maintained by canonical
WC pairs which constitute 73\% of all base-pairs.

\begin{table}[htbp]
\begin{center}
\begin{tabular}{|c|c|c|c|c|c|c|c|c|}
\hline
Helical context & \multicolumn{8}{c|}{Base-pair} \\
\hline
 & All & G$\cdot$C$_{\text{WC}}$ & A$\cdot$U$_{\text{WC}}$ &
G$\cdot$U$_{\text{w}}$ & G$\cdot$A$_{\text{s}}$ &
A$\cdot$U$_{\text{H}}$ & G$\cdot$A$_{\text{WC}}$ &
U$\cdot$U$_{\text{w}}$  \\
\hline
\multicolumn{9}{|c|}{Interior} \\
\hline
Intact &  0.62 & 0.62 & 0.75 & 0.63 & 0.34 & 0.05 & 0.25 & 0.74 \\
Nick   &  0.20 & 0.20 & 0.16 & 0.26 & 0.25 & 0.29 & 0.66 & 0.13 \\
Insert &  0.02 & 0.01 & 0.01 & 0.00 & 0.00 & 0.42 & 0.01 & 0.03 \\
\hline
\multicolumn{9}{|c|}{Ends} \\
\hline
Intact &  0.13 & 0.15 & 0.07 & 0.10 & 0.33 & 0.05 & 0.06 & 0.10 \\
Insert &  0.02 & 0.01 & 0.01 & 0.01 & 0.05 & 0.19 & 0.02 & 0.00 \\
\hline
\end{tabular}
\caption{Distribution in helical context of the seven most abundant
  base-pairs in our RNA helical regions dataset.}
\label{tab:helcontext}
\end{center}
\end{table}

\section{Deformability of Base-Pairs}
Figure   \ref{fig:pairs}  shows   a  visual   representation   of  the
deformability of the seven  most predominant base-pairs in RNA helical
regions. The overlapping structures for each base-pair are centered in
the  middle base  triad  (see appendix  \ref{appendix_1a} detail).  In
Table  \ref{tab:bppar}  we  have  collected  the  averages  and  their
corresponding  standard deviations  for the  rigid-body  parameters of
base-pairs, along with a
deformability score,  which is obtained  by the product of  the square
roots  of  the  eigenvalues  of the  base-step  parameters  covariance
matrix, such  volume score is  refered to as  accessible conformational
volume, also the root-mean-square deviation (rmsd) of the superimposed
structures in given.

The first  observation which stands out from  Table \ref{tab:bppar} is
that  non-canonical  base-pairs   are  clearly  more  deformable  than
canonical WC base-pairs.   This larger deformability in non-canonicals
comes mainly  from the in-plane  parameters, that is,  Shear, Stretch,
and Opening, whereas the  out-of-plane parameters Stagger, Buckle, and
Propeller are somewhat  comparable to those of the  less deformable WC
base-pairs, that is, there  is more deformability in non-canonicals at
the level of  hydrogen-bonding interactions, and less at  the level of
stacking interactions which constrain  the base-pairs to planarity. In
accordance with  this, the  variability of hydrogen-bond  distances is
also   greater  in   non-canonical   base-pairs  as   seen  in   Table
\ref{tab:seven}. Another  fact that confirms  this greater variability
is  that the  fraction  of  base-pairs which  comply  strictly to  the
hydrogen bonding pattern of  Table \ref{tab:seven} is much smaller for
non-canonicals as seen  in the last column. For  example, if one looks
at the amount of canonical  WC G$\cdot$C which complies to the binding
pattern shown  as given in  column three, then  one has 90  percent of
G$\cdot$C's  binding in  such pattern,  whereas  for the  case of  the
U$\cdot$U wobble  base-pair, only  54 percent of  base-pairs associate
following the  O2$\cdots$H-N3 and N3-  H$\cdots$O4 interacting pattern
strictly.    What's  occurring   with  the   additional  non-canonical
base-pairs  which  are  not  included in  the  hydrogen-bond  distance
averages  is that  they  are  either ``melting'',  that  is, they  are
forming less  hydrogen bonds  in order to  pair, or, they  are forming
different types of  heavy atom bridges mediated by  hydrogen. The fact
of  combining  units  of   degrees  and  angstroms  in  the  base-pair
parameters makes  the analysis of the conformational  volume score not
straightforward.   Nonetheless   the  general  trend   gives  a  clear
indication  of three main  levels of  deformation, that  is, canonical
G$\cdot$C and A$\cdot$U are not  very deformed in this space, having a
score around  six. G$\cdot$U$_{\text{w}}$, G$\cdot$A$_{\text{s}}$, and
A$\cdot$U$_{\text{H}}$ have all values  around 25, are definitely more
deformed    than    the   canonical's    but    not    as   much    as
U$\cdot$U$_{\text{w}}$    and    G$\cdot$A$_{\text{WC}}$.    A    more
straightforward  interpretation of  spatial  deformability comes  from
root mean square deviation (rmsd) values. These values are computed by
reconstructing  all atom  base-pair structures  using  their base-pair
parameters.   The  rmsd values  are  then  obtained  from an  all-atom
superposition  of  base-pairs  aligned  with  respect  their  standard
base-pair reference frames and then the average rmsd value is computed
with respect to the mean structure. The rmsd values have been obtained
using the  vmd software package  \cite{eargle2006}.  The deformability
of  base-pairs  as  ranked  by   their  rmsd  values  is  as  follows:
U$\cdot$U$_{w}$  (0.74 \AA)  > G$\cdot$A$_{WC}$  (0.54  \AA) $\approx$
G$\cdot$A$_{s}$  (0.50  \AA) >  G$\cdot$U$_{w}$  (0.45 \AA)  $\approx$
A$\cdot$U$_{H}$  (0.43 \AA)  > G$\cdot$C$_{WC}$  (0.38  \AA) $\approx$
A$\cdot$U$_{WC}$ (0.36 \AA)  and as it can be seen  it does not follow
exactly the same trend as  that of the conformational volume score, it
does nonetheless roughly maintain the  order of having three levels of
deformability,  although  in  this  case  the  G$\cdot$A$_{\text{WC}}$
base-pair   perhaps   should   be   grouped  with   the   intermediate
deformability non-canonical steps.  An interesting observation is that
the   base-pair  with   the  biggest   deformability,  that   is,  the
U$\cdot$U$_{\text{w}}$ base-pair is more  prominent in the interior of
intact  helical regions,  while  most of  the intermediate  deformable
non-canonicals show a preference for nicked or inserted regions in RNA
helical regions.

\begin{sidewaystable}[htbp]
\begin{center}
\begin{tabular}{|c|c|c|c|c|c|c|c|c|}
\hline
Base-pair & \multicolumn{8}{c|}{Rigid-body parameters} \\
\hline
 & Shear (\AA) & Stretch (\AA) & Stagger (\AA) & Buckle (deg) &
Propeller (deg) & Opening (deg) & V (deg$^{\text{3}}$
\AA$^{\text{3}}$) & rmsd \\
\hline
G$\cdot$C$_{\text{WC}}$ & -0.20 (0.41) & -0.15 (0.17) & -0.04 (0.40) & -3.4 (8.4)  &  -8.7  (8.5) &  0.5  (4.5)  & 6.5   & 0.38\\
A$\cdot$U$_{\text{WC}}$ &  0.04 (0.34) & -0.14 (0.15) &  0.04 (0.39) & -0.3 (8.4)  &  -9.0  (8.7) &  0.9  (5.2)  & 6.4   & 0.36\\
G$\cdot$U$_{\text{w}}$  & -2.11 (0.84) & -0.52 (0.27) & -0.04 (0.43) & -0.1 (8.2)  &  -7.4  (7.5) & -0.6  (6.9)  & 25.1  & 0.45\\
G$\cdot$A$_{\text{s}}$  &  6.78 (0.23) & -4.40 (0.55) &  0.14 (0.52) &  1.5 (11.0) &  -3.2  (9.1) & -5.3  (8.2)  & 27.2  & 0.50\\
A$\cdot$U$_{\text{H}}$  & -4.06 (0.80) & -1.92 (0.84) &  0.07 (0.61) & -0.4 (7.3)  &   1.0 (10.8) & -95.1 (17.4) & 25.9  & 0.43\\
G$\cdot$A$_{\text{WC}}$ & -2.34 (0.59) & -1.63 (0.31) & -0.09 (0.47) &  0.6 (8.8)  & -11.1  (7.8) & -0.2  (17.4) & 86.2  & 0.54\\
U$\cdot$U$_{\text{w}}$  &  0.00 (0.64) &  1.52 (0.40) & -0.29 (0.41) &  7.8 (10.8) & -10.8  (9.6) & -17.2 (13.5) & 49.8  & 0.74\\
\hline
\end{tabular}
\caption{Rigid-body base-pair parameters, their standard deviations
  and their conformational accessible volume score.}
\label{tab:bppar}
\end{center}
\end{sidewaystable}

%\section{Clustering of Yurong's Classification}

\bibliography{biblio}

