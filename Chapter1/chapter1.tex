%2345678901234567890123456789012345678901234567890123456789012345678901234567890
%%%%%%%%%%%%%%%%%%%%%%%%%%%%%%%%%%%%%%%%%%%%%%%%%%%%%%%%%%%%%%%%%%%%%%%%%%%%%%%%
% MAURICIO ESGUERRA NEIRA
% Ph. D. Thesis
%%%%%%%%%%%%%%%%%%%%%%%%%%%%%%%%%%%%%%%%%%%%%%%%%%%%%%%%%%%%%%%%%%%%%%%%%%%%%%%%
\chapter{Introduction}
\label{introduction} 
\bibliographystyle{nar}
%\section{RNA}
RNA plays  a primordial role  in life, and  perhaps also in  the early
history   of  its   origins  \cite{woese1967,   crick1968,  orgel1968,
  orgel2004}. In Biology RNA is  a central player in the transcription
and translation steps of what is known as its central dogma, i.e., DNA
makes  RNA   (via  transcription)   and  RNA  makes   protein  (during
translation).
%A  first RNA  step  transcribes  the genetic  message
%written in the DNA alphabet, to the RNA alphabet producing mRNA.
In  the  last   decade  of  the  twentieth  century   Fire  and  Mello
\cite{fire1998} found that RNA also plays a role previously thought to
be the  job of proteins. That  is, RNA can  regulate translation using
non-coding  RNA's (ncRNA's). Another  fundamental discovery  about RNA
came  in  2000  with the  elucidation  at  atomic  level detail  of  a
non-coding   RNA,    the   ribosome   \cite{schluenzen2000,   ban2000,
  wimberly2000}.

Since its very beginnings,  structural understanding of RNA has proven
to be a very complex problem. It was not until 1956, three years after
the famous Nature triad of papers by Watson and Crick, Wilkins, Stoke,
and  Wilson,  and  Franklin  and  Gosling  \cite{watson1953a}  on  the
double-stranded structure of DNA, that Alex Rich and David Davies were
able  to  produce   double-stranded  RNA  from  polyriboadenylic  acid
(poly-rA)  and polyribouridylic  acid  (poly-rU) to  produce a  neatly
difracting X-ray pattern typical of a double-helical structure. It was
not  until 1965 that  Robert Holley  was able  to obtain  the complete
sequence of yeast Alanine tRNA,  and also its secondary structure from
cleavage of  the whole  structure into smaller  fragments, and  it was
only in 1973,  that the first complex, but  small, tRNA structure, was
solved at full atomic detail.  Fifty seven years have passed since the
description  of the  double-helical structure  of DNA,  but  still RNA
faces more challenges with the  possibility of finding a whole new zoo
of non-coding RNA  structures \cite{weinberg2009}, and the possibility
of new engineered ones \cite{severcan2009}.

\section{RNA chemistry}
RNA is  a poly-nucleotide  chain, that is,  a polymer  whose monomeric
unit  is the  nucleotide.  The  nucleotide unit  is composed  of three
chemically distinct entities; base, sugar, and backbone. The bases can
be of two  types, purines, i.e. adenine and  guanine, and pyrimidines,
i.e. cytosine and uracyl as shown in Figure~\ref{fig:chemistry1}.
\begin{figure}
\centering
\includegraphics[scale=3.2]{Chapter1/chemistry1b.png}
\caption{A  single strand  of  RNA drawn  in  the 5$'$  to 3$'$  sense
  showing the  main chemical entities  which compose it;  base, sugar,
  and backbone.  The four bases (A,  G, C, U) are colored according to
  the  NDB  (Nucleic  Acid  Database)  convention  \cite{ndburl},  the
  backbone is colored gray, and the  sugars black. The bases G, and C,
  and the  furanose sugar  are numbered according  to the  IUPAC rules
  \cite{iupac1983}. This  figure is a  reproduction of Figure  2.1, in
  Wolfram Saenger's book \cite{saenger1984}}
\label{fig:chemistry1}
\end{figure}  

Heterocyclic bases can form a diversity of base-pairs through hydrogen
bonding and  can be classified in  28 classes as  suggested by Saenger
\cite{saenger1984},   and  as   seen   in  Figure~\ref{fig:saenger28}.
Nomenclatures which  conform to Saenger's classes  have been developed
by Lee-Gutell \cite{lee2004}, Leontis-Westhof \cite{leontis2002b}, and
Lemieux-Major \cite{lemieux2002}.

\begin{figure}
\centering
\includegraphics[scale=4.0, angle=90]{Chapter1/saenger28b.png}
\caption{Saenger  base-pairing  classes,  reproduced  from  his  book,
  "Principles of Nucleic Acid Structure". \cite{saenger1984}.}
\label{fig:saenger28}
\end{figure}  

The other  non-covalent interaction which is common  to the nucleotide
bases  is  that  of  stacking  through London  dispersion  forces  and
electrostatic  interactions.   It  has  been hypothesized  that  $\pi$
electron  interactions  could  also  account for  stacking,  but  very
precise  quantum calculations  have  show otherwise  \cite{sponer1996,
  sponer1997}.

Sugar, and  backbone can adopt a variety  of conformations constrained
to the values of their torsion  angles. In the case of the sugar these
torsion angles are called puckers,  perhaps in analogy to the gestures
made  by  human  lips.  Standards  to  describe the  conformations
resulting from the specific sets  of torsion angle values which sugars
and  backbone  can attain  have  been developed  and  can  be seen  in
textbooks  \cite{saenger1984}, in  the web  \cite{jenaurl}, and  in the
IUPAC recommendations  \cite{iupac1983}. We refer the  reader to these
sources for a  more detailed description, and limit  ourselves to show
in Figure~\ref{fig:puckersbbone} a  brief description of these torsion
angles.



A-RNA 
Loops

Reference:
Pure \& Appl. Chem., Vol.55, No.8, pp.1273—1280, 1983

\url{http://www.fli-leibniz.de/ImgLibDoc/nana/IMAGE_NANA.html}


\section{Standard reference frame and local parameters}
For  the  structural  description  of RNA  the  program  \textsf{3DNA}
\cite{lu2003} has been used.  In \textsf{3DNA} there are three sets of
symmetric local parameters:
\begin{enumerate}
\item Base-pair parameters,
\item Base (base-pair) step parameters,
\item Base (base-pair) local helical parameters.
\end{enumerate}
Bases or base  pairs are treated as ``rigid  bodies'' using ideas from
classical mechanics.
%% The first two are based on cartesiancoordinates
%% (``objectcentered'', reference frame),
%% whereas the third is referred as helical coordinates
%% (non-object centered)
The first two  sets of parameters are based  on Cartesian coordinates,
whereas the  third set  of helical coordinates,  resembles cylindrical
coordinates and is based on Chasles's theorem \cite{babcock1994}.

\subsection{Base-pair and base-step parameters}
In  \textsf{3DNA}   one  starts  with   a  Protein  Data   Bank  (PDB)
\cite{berman2000}  file   which  is  usually   based  on  experimental
information\footnote{This is the most common  case but the PDB file is
  sometimes  the result of  theoretical modeling.}   and which  can be
downloaded from the NDB or  PDB. This file contains the experimentally
obtained Cartesian  coordinates of  the atoms. With  this experimental
data one  performs a least-squares  fit to a standard  reference frame
\cite{olson2001}. This can be done using the \textsf{octave} script at
\url{http://rutchem.rutgers.edu/\~{}esguerra/RNA/scripts.html}   as  a
tutorial  example.  The  coordinate origin  which is  embedded  in the
standard  reference frame  is kept  and used  for both  base  and base
pairs.  In  the case of single  unpaired bases, the  program keeps the
origin of one  base of an ideal Watson-Crick  pair.  The definition of
this frame is illustrated in Figure~\ref{fig:standard}

\begin{figure}[htbp]
\centering
\includegraphics[scale=0.8]{Chapter1/standard.png}
\caption{Standard   reference   frame  of   an   A-T  base-pair.   The
  \textit{y}-axis (dashed green line) is  chosen to be parallel to the
  line connecting the C1$^{'}$ of  adenine and the C1$^{'}$ of thymine
  associated in  an ideal Watson-Crick  base-pair. The \textit{x}-axis
  is the perpendicular  bisector of the C1$^{'}$ -  C1$^{'}$ line, and
  the origin is located at the intersection of the \textit{x}-axis and
  the  line connecting  the C8  atom  of adenine  and the  C6 atom  of
  thymine. The  \textit{z}-axis is the cross product  of the $\hat{x}$
  and $\hat{y}$ unit vectors.}
\label{fig:standard}
\end{figure}

Once one has determined the coordinate origins for the two consecutive
bases or base pairs comprising a  step one defines a middle step triad
(MST) \cite{lu1997}. This can be described by the following procedure:

1) Find the angle $\Gamma$ between consecutive normals, \textit{i.e.},
\textit{z}-axis. Since these are unit vectors, the angle is defined by
the dot product:

\begin{gather}
\Gamma = \cos^{-1} (\hat{z}_i \cdot \hat{z}_{i+1})
\end{gather}

2)  Then find the  vector which  is perpendicular  to the  two normals
(\textit{z}-axis). This  vector is obtained from the  cross product of
the  consecutive \textit{z}-axis  (that is,  the normal  to  the plane
formed by the two vectors). This axis is called the roll-tilt axis and
is normalized to form the unit vector $\hat{r_t}$,

\begin{gather}
\hat{r_t} = \frac{\hat{z}_i \times \hat{z}_{i+1}}{|\hat{z}_i \times
\hat{z}_{i+1}|}
\end{gather}

%% Note: Why do you need a matrix? Isn't this always in 3D and therefore
%% just vectors would do? Why the more general matricial way instead of
%% the vectorial representation?
3) To make consecutive \textit{z}  vectors coincide, one uses a linear
homogeneous transformation  $R(\theta)$ about the  roll-tilt axis such
that the original orientation matrices $T_i$ and $T_{i+1}$ are rotated
by $ \theta  = \pm \Gamma / 2$ to yield  the transformed $T_i^{'}$ and
$T_{i+1}^{'}$ orientation matrices.
%% NEED to include GRAPHICS of ANGLES and COORDINATES in general.

\begin{gather}
T_i^{'} = R_{rt}(\pm \Gamma/2) T_{i} \\
T_{i+1}^{'} = R_{rt}(\mp \Gamma/2) T_{i+1}
\end{gather}

The origin  for the middle step  triad is the average  of the position
vectors for the $i$ and $i+1$ reference frames,

\begin{gather}
r_{MST} = \frac{(r_i + r_{i+1})} {2}
\end{gather}

4) Again  using the  dot product  one can find  the angle  between the
transformed  $\hat{y}^{'}$  vectors.   This  angle  is  equal  to  the
magnitude  of   the  Twist  ($\Omega$).    The  dot  product   of  the
$\hat{z}_{MST}$ unit  vector with the vector resulting  from the cross
product of $\hat{y}_{i}^{'}$ and $\hat{y}_{i+1}^{'}$ gives the sign of
$\Omega$. Since  the transformed  \textit{x-y} plane is  orthogonal to
$\hat{z}$ then this applies in the same manner for \textit{x},
%%The  vector resulting  from (yi  X yi+1  has got  to be  parallel or
%%anti-parallel to zMST, there's no other possibility.

\begin{gather}
\Omega = cos^{-1}(\hat{y}_{i}^{'} \cdot \hat{y}_{i+1}^{'})\\
(\hat{y}_{i}^{'} \times \hat{y}_{i+1}^{'}) \cdot \hat{z}_{MST} > 0, \quad \textrm{then} \ \Omega > 0\\
(\hat{y}_{i}^{'} \times \hat{y}_{i+1}^{'}) \cdot \hat{z}_{MST} < 0, \quad \textrm{then} \ \Omega < 0
\end{gather}

%%if normalized beforehand then the rule would be,
%%\begin{gather}
%%(\hat{y}_{i}^{'} \times \hat{y}_{i+1}^{'}) \cdot \hat{z}_{mst} = 1, \quad %%then \ \Omega > 1\\
%%(\hat{y}_{i}^{'} \times \hat{y}_{i+1}^{'}) \cdot \hat{z}_{mst} =
%%-1,\quad then \ \Omega < -1
%%\end{gather}

5) With more scalar product one  can find other angles, such as the phase
angle $\phi$,
%% Show in figure.

\begin{gather}
\phi = cos^{-1}(\hat{rt} \cdot \hat{y}_{MST})\\
(\hat{rt} \times \hat{y}_{MST}) \cdot \hat{z}_{MST} > 0, \quad \textrm{then} \ 180 \geq \phi \geq 0\\
(\hat{rt} \times \hat{y}_{MST}) \cdot \hat{z}_{MST} < 0, \quad \textrm{then} \ -180 \leq \phi \leq 0
\end{gather}

6) The  roll $\rho$  and tilt $\tau$  angles, which are  the remaining
angular degrees of  freedom for step parameters, are  defined in terms
of the bending angle and the phase angle:
%as if we were doing a change of variables from cartesian to polar.

\begin{gather}
\rho = \Gamma cos (\phi)\\
\tau = \Gamma sin (\phi)
\end{gather}

Now to  get the remaining  three translational degrees of  freedom for
step  parameters  ($Dx,  Dy,  Dz$)  one  just  needs  to  express  the
displacement vector in the middle step triad frame:

\begin{gather}
[D_xD_yD_z]=T_{MST}(r_{i+1} - r_{i})
\end{gather}

The  procedure  is  completely  analogous  to  compute  the  base-pair
parameters.   The  opening $\omega$,  buckle  $\kappa$, and  propeller
$\sigma$  are the  analogs of  twist $\Omega$,  roll $\rho$,  and tilt
$\tau$, and  the middle  step triad is  called middle base  triad MBT.
The axis  which are made to  coincide are the  \textit{y}-axis and not
the \textit{z}-axis as in the base-pair step case \cite{lu1997}.

The parameters obtained by  this procedure are depicted graphically in
Figure~\ref{fig:allparam}.
\begin{figure}[htbp]
\centering
\includegraphics[scale=0.6]{Chapter1/allparam2.png}
\caption{Illustration of base pair and base step parameters \cite{lu2003}}
\label{fig:allparam}
\end{figure}


%\begin{gather}
%\gamma = \cos^{-1} (\hat{y}_{iII} \cdot \hat{y}_{iI})\\
%bo = \hat{y}_{iII} \times \hat{y}_{iI}\\
%\hat{bo} = \frac {bo}{\sqrt{bo \cdot bo}}\\
%T_{iII}^{'} = R_{bo}(+\frac{\gamma}{2}) T_{iII} \\
%_{iI}^{'} = R_{bo}(-\frac{\gamma}{2}) T_{iI}\\
%r_{mbt} = \frac{(r_{iII} + r_{iI})} {2}\\
%\omega = cos^{-1}(\hat{x}_{iII}^{'} \cdot \hat{x}_{iI}^{'})\\
%(\hat{x}_{iII}^{'} \times \hat{x}_{iI}^{'}) \cdot \hat{y}_{MBT} > 0, \quad then \ \omega > 0\\
%(\hat{x}_{iII}^{'} \times \hat{x}_{iI}^{'}) \cdot \hat{y}_{MBT} < 0, \quad then \ \omega < 0\\
%\phi^{'} = cos^{-1}(\hat{bo} \cdot \hat{x}_{MBT})\\
%(\hat{bo} \times \hat{x}_{MBT}) \cdot \hat{y}_{MBT} > 0, \quad then \ 180 \geq \phi^{'} \geq 0\\
%(\hat{bo} \times \hat{x}_{MBT}) \cdot \hat{y}_{MBT} < 0, \quad then \ -180 \leq \phi^{'} \leq 0\\
%\kappa = \gamma cos (\phi^{'})\\
%\sigma = \gamma sin (\phi^{'})\\
%[S_xS_yS_z]=T_{MBT}(r_{iI} - r_{iII})
%\end{gather}

\subsection{Local helical parameters}

Local helical parameters are determined using Chasles's theorem, which
states \cite{babcock1994}:
\begin{quote}
``\textit{One can  always transport  a free rigid  body from one  position and
  orientation  to  another  position   and  orientation  by  a  single
  continuous motion along a unique axis of rotation.}''
\end{quote}

\noindent For  the three dimensional  case of nucleic acid  base steps
what   this  means   is   that,  instead   of   rotating  around   one
reference-frame  centered  axis  and  then translating  along  another
reference-frame centered  axis, one rotates about  and also translates
along   only   one  common   axis,   which   is  not   reference-frame
centered. This allows one to  define the orientation of a helical axis
(or unique  rotational-translational axis) as  a unit vector  given by
equation 2.15:

\begin{gather}
h=\left[ \begin{array}{c}
h_x\\
h_y\\
h_z
\end{array} \right]
\end{gather}
where:
\begin{gather}
h_x = \frac{\tau}{\Omega_h}, \qquad h_y = \frac{\rho}{\Omega_h},
\qquad h_z = \frac{\Omega}{\Omega_h}
\end{gather}

\begin{gather}
\Omega_h = \sqrt{\tau^2 + \rho^2 + \Omega^2}
\end{gather}

The local helical axis can be defined alternatively \cite{bansal1995}
as a cross product:

\begin{gather}
h = (x_2 - x_1) \times (y_2 - y_1)
\end{gather}
where the  $x$ and $y$ refer to  the reference frames on  base pairs 1
and 2.

\section{RNA folding}
The first  high-resolution X-ray\index{X-ray} structure  of RNA larger
than a dinucleotide was  that of yeast tRNA$^{\textrm{Phe}}$ at 3{\AA}
in  1974 \cite{robertus1974,  kim1974, stout1976}.   Thirty  six years
later there  are two orders  of magnitude more  structural information
about RNA \cite{noller2005}, and new information from non-coding RNA's
is  expected  \cite{weinberg2009}.  This  fact  and  the discovery  of
ribozymes  \cite{kruger1982,  takada1983},  which  are  catalytic  RNA
molecules, has  renewed interest in solving  the RNA folding\index{RNA
  folding}  problem,  that is,  starting  from  the primary  sequence,
finding in  an automated\footnote{The term  automated is used  here to
  mean  a  theoretical model  of  tertiary  folding,  which could  use
  experimental measures of secondary structure association in the same
  way   that  the  traditional   secondary  structure   folding  model
  \cite{zuker1989,    hofacker1994}    uses    the    Tinoco-Uhlenbeck
  dinucleotide   postulate  \cite{borer1974}   to   find  total   free
  energies.}   way the  native three-dimensional  structure of  an RNA
molecule  and   the  folding  pathway   that  it  follows.    The  RNA
folding\index{RNA folding} problem is usually seen as analogous to the
protein folding  problem, due both  to the discovery of  the enzymatic
behavior  of  RNA \cite{kruger1982,  takada1983}  and the  complicated
folding of large RNA molecules \cite{batey1999}.  To take advantage of
this analogy,  a unified conceptual  framework for describing  RNA and
protein folding, called the  kinetic partitioning mechanism (KPM), has
been developed by Thirumalai and Hyeon \cite{thirumalai2005}. This and
other methods are based on defining an adequate partition function for
describing  the correct conformational  ensemble of  folded, partially
folded,    and   unfolded    structures    \cite{chen1995,   chen1998,
  thirumalai1996} of either protein or RNA.

\section{Is RNA folding a hard or easy problem?}
There  are  two trains  of  thought  regarding  the mechanism  of  RNA
folding.  One  states that  RNA folding is  less complex  than protein
folding  \cite{tinoco1999} because  RNA is  made up  of a  four letter
alphabet of similar  nucleotide units instead of a  20 letter alphabet
of  dissimilar   amino  acids.   Therefore  the   number  of  possible
sequential  combinations  is smaller.   It  is  also  well known  that
secondary and  tertiary interactions can  be separated in the  case of
RNA  by the absence  or presence  of Mg$^{2+}$  \cite{rangan2003} (see
Figure~\ref{fig:folding}), and  that the secondary structure motifs
of RNA are more limited  in  number  than  those  of protein,  whereas
secondary  and tertiary elements are not as easily separable in
proteins.
\begin{figure}[ht]
\centering
\includegraphics[scale=0.3]{Chapter1/rangan2003pnas.png}
\caption{Separation  of  secondary  and  tertiary interaction  in  RNA
  \cite{rangan2003}. Double helical secondary structure represented by
  individual  cylinders and  tertiary interactions  by  association of
  cylinders. Color coding stands for separate  helical regions of
  RNA, and the connecting  black strings represent single stranded loop
  structures.}
%% Note that Dr. Olson asks what the colors in cylinders mean.
%% Answer: They mean nothing. Perhaps only that the cyan stands for
%% the helical structures making a pseudo-knot, that is, P7 and P3.
\label{fig:folding}
\end{figure}
The  other point of  view says  that RNA  folding can  be at  least as
complex as protein folding \cite{moore1999a, sorin2004} since there is
no such thing  as hydrophobic burial of regions of RNA  as in the case
of  proteins.   Instead, the  electrostatic  problem  stemming from  a
complex charged backbone must be dealt with in the case of RNA.
% The case of the electrostatic treatment of the backbone is lacking
% here, most likely WKO wants me not to ignore our own Gerald Manning
% tinoco 1999 says this must be an easy to solve problem since we can
% do the electrostatics for it easily
For instance,  the interactions of  the RNA polyanionic  backbone with
water  and cations  \cite{klein2004a}  are not  easily simulated  with
explicit   solvent  models   like those used to treat   proteins.  The
aforementioned interactions of RNA  need to be modeled implicitly, and
must aim to describe long dynamic processes of the order of seconds to
minutes,  in  contrast   to  the  typical  time  scales   of  tens  of
microseconds associated with protein folding.
% Remember that this means that a explicit calculation for RNA would
% be prohibitively large.

Although   secondary   and  tertiary   structure   can  be   separated
experimentally, there have been few theoretical efforts to account for
the  folding  of  RNA  from  a random  sequence  of  nucleotides  into
secondary structures and tertiary  structures. What little is know has
been  investigated at  low resolution.  Stephen Harvey  and associates
have   simulated   the   folding   of   yeast   tRNA$^{\textrm{Phe}}$,
\cite{malhotra1990}  and  the  assembly  of  the 30S  subunit  of  the
ribosome \cite{stagg2003} at various levels of detail, initially using
only one pseudoatom  per helical region, and later  one pseudoatom per
nucleotide.  Recently  Fran\c{c}ois  Major's  group  at  Montreal  has
proposed a pipeline of two  computer algorithms to study RNA structure
\cite{parisien2008}.   One    pipeline   makes   secondary   structure
predictions, and the  other assembles 3D structures based  on the best
scoring secondary structures.
% Note for presentation ==> Include figure 1 in Malhotra-harvey paper
% Look at what says in folding.stanford.edu/science.html Also take
% into account Biophys J. V.88 2516-2524 for the case of having to
% think of water in the folding problem.
By contrast,  in the case of  proteins many groups  have simulated the
transition  from  secondary  to  tertiary  structure,  including  some
calculations which  account for the  strong coupling of  secondary and
tertiary  structure   \cite{westhead1999,  gerstein2003,  meiler2003}.
This type of work is  often referred to as protein structural topology
and there is no counterpart for RNA.

%The seminal paper here seems to be the one of liphardt in 2001 for
%rna unfolding
\section{Experimental folding techniques}
Traditionally   RNA   folding  and   unfolding   have  been   followed
calorimetrically  and spectroscopically as  a function  of temperature
and cation concentration  \cite{bloomfield2000, boots2008}. While this
approach  works well  for studying  two-state  folders, \textit{i.e.},
structures  which populate  only two  states (native  and  melted), in
general RNA's  are not  two-state folders. RNA  seems to go  through a
rugged  free  energy landscape  of  conformations  in  the process  of
folding \cite{zhuang2003}.  The  experimental solution to this problem
is offered  by single-molecule techniques  like fluorescence resonance
energy transfer (FRET) and  mechanical micromanipulation, in which the
ends of RNA  are attached to micron sized beads  which are then pulled
apart  and  monitored  with  a laser  light  trap  \cite{liphardt2001,
  onoa2004,  tinoco2004, hyeon2005}.  In  the case  of single-molecule
force-induced   unfolding,  state   transitions   often  occur   under
non-equilibrium  conditions, thereby  making it  difficult  to extract
equilibrium  information from the  data. Bustamante, Tinoco,
and associates have shown that by using the Crooks fluctuation theorem
\cite{crooks1999},  one  can deal  with  such  cases  and extract  RNA
folding    free     energies    from    single-molecule    experiments
\cite{collin2005}. Recently an alternative solution to this problem has
been proposed by Thirumalai and associates based on single-molecule
force-quenching experiments, by using a so called  de Genes "expanding
sausage model" \cite{hyeon2009}.
%\subsection{RNA Folding in Vitro vs in Vivo vs in Silico}
% It still is to be seen whether the following is relevant or not
% This single molecule information is
% collected in-vitro and not in-vivo, which is actually the ultimate
% problem aimed for prediction, there's quite a lot of evidence for
% different folding states reached in one case and not the other and
% viceversa \cite{sosnick2003, schroeder2002} but still a first step
% towards understanding in-vivo folding is in-vitro and in-silico
% experimentation.


\section{RNA simulations}
Network  and molecular  mechanics-molecular  dynamics (MM-MD)  methods
provide  useful  information  relevant  to the  RNA  folding-unfolding
problem, especially  for describing fluctuations away  from the native
conformation.   Gaussian network  models  \cite{y_wang2004, bahar1998,
  wang2005}, which  treat RNA  at less than  atomic detail,  have been
used  to  describe  the  motions  of large  RNA  structures  like  the
ribosome.  Examples  of the  predicted normal modes  of motion  of the
ribosome  can be  seen at:  http://ribosome.bb.iastate.edu/70SnK mode.
Using MM, Sanbonmatsu and coworkers  obtained a static atomic model of
the    70S    ribosome    structure    through    homology    modeling
\cite{tung2004}.  Tung  and  associates  used this  structure  for  an
all-atom  MD simulation  of the  movement of  tRNA into  a fluctuating
ribosome  \cite{sanbonmatsu2005}.  This  type of  simulation  might be
useful in a  reverse-folding approach to the RNA  folding problem.  To
the best of our knowledge,  such calculations haven't as yet been done
for RNA.

\subsection{Local nucleotide interactions}
%\subsubsection{QM approaches and MM consequences}
The molecular  interactions that  rule  RNA structures at  the nucleic
acid base level, \textit{i.e.},  local level, are hydrogen bonding and
stacking interactions. The former are  related to base pairing and the
latter, in most cases, to  nucleotide steps. These interactions can be
explored  theoretically at various  levels. At  the highest  level are
ab-initio  quantum   mechanical  calculations  which   are  still  too
expensive   for  systems  as   large  as   hundreds  of   atoms.  Such
calculations,  nevertheless,  can  tell   a  great  deal  about  local
electronic behavior.  For example, Hobza and  collaborators have found
that the  stacking interaction of free nucleotide  bases is determined
by   dispersion  attraction,   short-range  exchange   repulsion,  and
electrostatic  interaction.  No  specific $\pi-\pi$  interactions  are
found     from    electron    correlated     ab-initio    calculations
\cite{sponer1996, sponer1997}.  This is  why force field  methods have
been so successful in the  study of nucleic acids, since the empirical
potentials used  in such studies  mimic well the  quantum mechanically
obtained energy profiles \cite{tung2004, sponer2000}.
% since they can be modeled easily with simple empirical potentials
% consisting of Lennard-Jones, van der Waals and Coulomb terms.
% What the recent results say it's simply that by using a larger
% basis set, they can account for some interactions which were not
% included before, and maybe because of taking better account of
% electron-correlation.
A currently debated ab-initio finding is whether small fluctuations in
the   configurations   of   neighboring   base  pairs   (dimers)   are
iso-energetic  or  not.   Recent  calculations  of  Sponer  and  Hobza
\cite{sponer2006}    seem   to    contradict   their    earlier   work
\cite{sponer2000,  hobza2002},  in which  the  stacking energies  were
reported to  be relatively insensitive to dimer  conformation. The new
results  use  the so-called  ``coupled  cluster  singles doubles  with
triple electron excitations'' CCSD(T)  method, to account for electron
correlation.  Using  this electron correlation  energy correction, the
stacking energy differences between dimer conformations turn out to be
considerably higher than previously reported.

% therefore justifying rigid body parameter interpretations.
% \subsubsection{Experimental Stacking and Polyionic backbone}
Single-strand and double-strand stacking free energies can be obtained
calorimetrically \cite{freier1985}.   One of the  most popular methods
used   for  obtaining   such  quantities   is   differential  scanning
calorimetry (DSC) \cite{marky1982}.  These measurements show favorable
dinucleotide stacking  free energies as  large as $-3.6$  kcal/mol for
double-strand  stacking.   Experimentally,  the  magnitudes  of  these
interactions      are     found      to      be     sequence-dependent
\cite{bloomfield2000}. In  fact, the  stacking free energies  for some
sequences\footnote{Free   Energies   for   5'   unpaired   nucleotides
  (e.g. UC/A  UU/A) are quite small  (i.e.  $<$ 0.4  kcal/mol) and are
  termed  weakly stacking bases.\cite{burkard1999,  burkard1999b}} are
found to  be negligible.   Thus there may  be no  accountable stacking
interaction at all for some sequences.

Besides  taking into  account  the effects  of  stacking and  hydrogen
bonding,  it  is  important  to  think  at the  same  time  about  the
polyelectrolyte  nature  of the  RNA  backbone.  Manning's  counterion
condensation theory \cite{manning1977,  manning2003} provides a simple
and   quantitative  picture   of   the  interactions   of  a   regular
double-helical  nucleic acid  polyanion with  its counterions,  but it
does   not  take   into  account   the  discrete   nature   of  charge
\cite{bloomfield2000} or the  folding of RNA. Poisson-Boltzmann theory
offers a more detailed picture of the behavior of charged macroions in
solution \cite{antypov2005, xu2007}.
% Talk more about counterion condensation, thirumalai discusses it on
% his 2001 paper also chapter 8 of Bloomfield, Crothers, Tinoco
% (References \cite{manning2003} Ray-Manning?).
% Real Experiments results for stacking energies and polyanion
% energies and Energetics related to cation metal presence
% WKO says that there might be old experimental data that are contrary
% to this and that I must show it here, so far what I've found is
% Saenger saying that based on old QC and this is different, he
% relates it to hydrophobicity concepts. Talk about experiments.

The local conformational  space of RNA has been  studied using a large
set of available  RNA structures from the Nucleic  Acid Database (NDB)
\cite{berman1992}.  The  torsion angles  of the nucleotide  steps have
been   clustered    using   different   techniques   \cite{murray2003,
  schneider2004}.   The  root-mean-square  deviations  (RMSD)  of  the
distances between closely spaced  atoms in the phosphates, sugars, and
bases, have  also been clustered \cite{sykes2005}.  The latter studies
are aimed  at finding the  common nucleotide base steps  and base-pair
building  blocks which  have  been  given the  name  of RNA  doublets.
Recently, the  RNA Ontology Consortium (ROC) has  proposed a consensus
set  of RNA dinucleotide  conformers integrating  the work  of various
groups \cite{richardson2008}.


\subsection{RNA  secondary  structure   algorithms  and  the  lack  of
 tertiary ones}
From   secondary   structure   prediction  algorithms   like   Zuker's
\textit{mfold} program \cite{zuker2003}, Hofacker's Vienna RNA package
\cite{hofacker1994}, or  Mathews Dynaling software \cite{mathews2002},
one obtains  a large ensemble  of secondary structure graphs,  i.e. 2D
representations of  the double-stranded helical  stems, hairpin loops,
bubbles formed by the constituent bases.  These graphs can be analyzed
with graph  theory to produce  a partition function describing  a full
arrangement  of contacts for  the total  number of  possible secondary
structures, allowing  the construction  of a "relation  of microscopic
conformations to macroscopic  properties" \cite{chen2000}. So far this
type of model  has not been generalized to  take into account tertiary
structural features, \textit{i.e.},  interhelical interactions of RNA.
In  the  last  two to  three  years  a  boom  in prediction  of  small
($\approx$ 200  nucleotides) RNA 3D structures  has started. Basically
three types of approaches are being  followed.  One is that of using a
coarse-grained model, assigning a potential function to it, applying a
minimization procedure, and then performing a molecular mechanics (MM)
all-atom  refinement \cite{das2007,  ding2008,  jonikas2009a}. Another
starts  from  the predicted  secondary  structures,  assumes that  the
helical  regions adopt  the canonical  A-form  structure, mechanically
inserts residues as rigid bodies in the remaining non-helical regions,
and  finally carry  out an  MM optimization  \cite{martinez2008}.  The
third  approach   entails  a  pipeline   between  secondary  structure
prediction, and tertiary structure assembly is proposed. This pipeline
uses  as bridging  concept  between  2D and  3D  structure, the  graph
theoretical definition of a minimum cycle basis, which for the case of
nucleic  acids  has  been  renamed  as  Nucleic  Cyclic  Motifs  (NCM)
\cite{parisien2008}.

\subsection{RNA  overall fold}
Whereas  in  the case  of  proteins  one  qualitatively describes  the
overall  fold  in terms  of  the  arrangement  of secondary  structure
motifs, \textit{i.e.}, using the helix-ribbon-coil images developed by
Jane           Richardson          \cite{richardson2000}          (see
Figure~\ref{fig:ribboncoil}), there is still no comparable description
of  the overall  fold of  RNA. A  ribbon representation  of  the sugar
phosphate backbone (see Figure~\ref{fig:ribosome}) helps to understand
the  folding of  small  RNA's, but  in  the case  of  the ribosome,  a
representation at such level of detail does not allow to make sense of
such  a large  structure  (close  to 3000  nucleotides  for the  large
subunit of  the archaeal  ribosome).  In the  past two  years Holbrook
\cite{holbrook2008}  and  Sykes  \cite{sykes2009}  have  proposed  new
representations for RNA based on helical region organization. Holbrook
makes  an  analysis of  continuous  interhelical  strands, so  called,
COINS, and Sykes makes an optmized projection of 3D helical axis to 2D
images,  which can  later  be annotated  with,  for example,  hydroxil
radical footprinting results.

\begin{figure}[ht]
\centering
\includegraphics[scale=0.4]{Chapter1/overallfold.png}
\caption{Ribbon-coil    schematic    illustraring    the   fold    and
  intermolecular  units of  a dimer  of prealbumin  (PDB\_ID:2pab), or
  transthyretin,    taken     from    Richardson    \textit{et    al.}
  \cite{richardson2002}}
\label{fig:ribboncoil}
\end{figure}

\begin{figure}[t]
\centering
\includegraphics[scale=0.5]{Chapter1/ribosome_ribozyme.png}
\caption{Images   of  the  \textit{Haloharcula   marismortui's}  large
  ribosomal subunit NDB\_ID:RR0033 (left) and the hammerhead ribozyme
  (right) NDB\_ID:UR0029.
  The  figures were taken directly  from the NDB
  web  pages,   and  show  a  3DNA   generated  \cite{lu2008b}  ribbon
  representation of the phosphate backbone, and a block representation
  for the nucleotide bases. From  the figures it's clear that, whereas
  the   ribozyme   fold   can   be  clearly   understood   with   this
  representation, the ribosome fold cannot.}
\label{fig:ribosome}
\end{figure}

One  can  envision that  a  thorough  investigation  of the  space  of
translational and rotational degrees of freedom of the helical regions
of RNA could give clues as to  how we might see an overall fold in RNA
structures.  To the  best  of  our knowledge  there  is no  comparable
quantitative description of the folding of proteins.

In  the  case  of  proteins  the SCOP  (Structural  Classification  of
Proteins)  database  \cite{andreeva2004},  classifies proteins,  among
various  qualitative  descriptors,   according  to  folds,  which  are
recurrent  arrangements of  secondary structure,  that is,  a  list of
secondary  structures with  unique topological  connections.  The SCOR
(Structural  Classification  of  RNA)  database  \cite{klosterman2002,
  klosterman2004}, aims  to provide  a similar classification  to that
obtained   for  proteins,   but   using  RNA   motifs  instead.   This
classification focuses on the local folding of small pieces of RNA and
cannot  describe  the  overall  fold.  Local  classification  is  also
qualitative rather than quantitative.

\subsection{RNA motifs}
The term ``\textit{RNA motif}'' is  used in the literature to describe
three   different   levels    of   RNA   organization,   namely,   RNA
\textbf{sequence} motifs, RNA  \textbf{secondary structure} motifs, or
RNA \textbf{3D structure} motifs.   Because these distinctions are not
always clearly made the beginner may result in confused and frustrated
bibliographical searches.

The lack of a unique definition of RNA motifs is yet another source of
confusion  in  understading  RNA  motifs  is  the  lack  of  a  unique
definition.  Three  popular and  somewhat  recent  definitions of  RNA
motifs include:
\begin{itemize}
\item{``\textit{a discrete sequence or combination of
    base  juxtapositions   found  in  naturally   occurring  RNA's  in
    unexpectedly high abundance.}''\cite{moore1999}}
\item{``\textit{conserved structural subunits that make
    up the secondary structures of RNAs.}''\cite{holbrook2005}}
\item{``\textit{ordered   stacked    arrays   of
    non-Watson-Crick  base  pairs  that  form distinct  folds  on  the
    phosphodiester backbones of RNA strands.}''\cite{leontis2003}}
\end{itemize}

The  kind of RNA  motifs addressed in this
thesis are of the third type, that is, RNA \textbf{3D structure}
motifs  which we henceforth term RNA  motifs.
From our point of view RNA motifs are to be understood as  peculiar
sets of geometrical  (in the rigid block sense)  arrangements in
three-dimensional space.

Even  though there  is no  unique definition,  we can  think  of three
practical  tasks regarding  RNA  motifs.   That is,  given  an RNA  3D
structure    automatically    identify,   describe    \cite{laing2009,
  laing2009a, holbrook2008,  spackova2006, reblova2003}, and  find new
\cite{sarver2008,  mokdad2008,  duarte2003,  stonge2007,  lemieux2006}
motifs.   For   automatic  identification  of  RNA   motifs  Pyle  and
collaborators have  developed a software called  AMIGOS. This software
finds RNA motifs based on specific values of backbone
virtual torsion  angles $\nabla$ and  $\theta$ \cite{yathindra, olson,
  duarte2003} in  a way which resembles a  Ramachandran plot analysis.
Lemieux  and Major  \cite{lemieux2006} provide  the  software MC-Fold,
which implicitly finds  RNA motifs based on an  algorithm to determine
so  called nucleic  cyclic motifs,  which are  just the  minimal cycle
basis  of an RNA  secondary structure  interpreted as  a mathematical
graph.   Leontis \cite{nasalean2009}  and  collaborators provide  FR3D
(read as  FRED).  F3RD  is a matlab  windows executable  program which
finds  RNA motifs based  on the  isostericity matrices  of base-pairs.

For description of RNA motifs Schlick and collaborators have used FR3D
to localize  RNA helical junctions of order  four (four-way junctions)
or higher, and have visually analyzed if the helices in such junctions
form   coaxial  stacks   or   not  and   clasified  them   accordingly
\cite{laing2009, laing2009a}.  As  mentioned previously in the context
of  RNA  folds, Holbrook,  and  Sykes,  describe  helical regions  and
display them in two-dimensional representations.


so called COINS and
provides their location in  secondary structure representations of the
large ribosomal subunit \cite{holbrook2008}.



order to produce a 3D structure from a sequence, which is based on their analysis of RNA
secondary structures  through a  graph theoretical method  which finds
the 





\section{Overview}
Keeping always in  mind the greater scope of  the RNA folding problem,
this thesis  addresses various issues of  RNA structural understanding
using RNA crystallographic data from the Protein Data Bank (PDB). Such
data  has been  analyzed statistically  in terms  of a  very rigourous
rigid-body formalism.  In Chapter 2 the consensus clustering technique
is   used  to   classify   RNA  base-step   parameters  of   non-A-RNA
conformations, and  the resulting groups are  localized and understood
in the context  of rRNA.  Chapter 3 reconsiders  previous work carried
out by  Dr. Yurong Xin  at the Olson's  lab, on classification  of RNA
base-pairs by  resorting again to clustering  analysis techniques, and
database  mining  of the  WWW  available  Base  Pair Structures  (BPS)
database.  In  Chapter 4 we  explore, using statistical  analysis, the
data available  on RNA  helical regions, and  use this  information to
compute the persistence length of double-stranded RNA's and compare it
to  experimental  results.  In  Chapter  5 we  provide  a  new  python
software,  pyRNAmotifs which interfaces  with 3DNA  to do  a rigourous
search of existing and perhaps  new RNA motifs, and finally in Chapter
6  we propose  the measurement  and classification  of  RNA structures
using a  new graph theoretical index  named folding index,  based on a
helical region "view"  of RNA's, which is clearly  concordant with the
emerging  necessity   of  new  metrics  beyond   RMSD  for  structural
understanding.

\bibliography{biblio}
